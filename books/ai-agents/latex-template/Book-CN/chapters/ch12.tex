\chapter{构建你的智能体技术栈}

\section{差点毁掉我生意的技术栈}

我记得我的自动化基础设施崩溃的那个夜晚。

那是凌晨2点,我被十七封愤怒的客户邮件、一个着火的Slack频道和一个已经宕机六小时的支付系统惊醒。罪魁祸首?一系列连锁故障:

\begin{itemize}
\item 我的工作流工具发生了意外宕机
\item 这触发了我的AI集成错误
\item 这破坏了我的客户数据库状态
\item 这导致向同一个客户发送了四十七次相同的发票
\end{itemize}

我构建的智能体技术栈就像一个鲁布·戈德堡机器。每个部件都依赖于每个其他部件。当一个多米诺骨牌倒下时,它们全都倒了。

第二天早上,疲惫而尴尬,我重新开始。这次,我为弹性而构建。为简单而构建。为那种让你晚上能睡好觉的可靠性而构建。

那个技术栈就是我将在这一章教你构建的。

\section{构建还是购买的决策}

在我们深入技术架构之前,让我们解决每个创始人面临的根本问题:你应该使用现成的AI工具还是构建定制智能体?

诚实的答案是两者都用,按顺序进行。

从现有工具开始。它们将以百分之十的努力覆盖你百分之八十的需求。只有当你遇到它们的限制时才构建定制——当变通方法变得比从头构建更复杂时,你就会知道你遇到了限制。

\begin{codebox}
\begin{lstlisting}[style=python]
BUILD CUSTOM WHEN:                  # 何时定制构建:
- Your process is genuinely unique  # 你的流程真正独特
- Integration costs exceed build costs
  # 集成成本超过构建成本
- You need complete control over behavior
  # 你需要完全控制行为
- Data privacy is non-negotiable   # 数据隐私不可妥协
- You have technical capacity (or budget for it)
  # 你有技术能力(或预算)

USE EXISTING TOOLS WHEN:            # 何时使用现有工具:
- Standard workflows (email, calendar, CRM)
  # 标准工作流(邮件、日历、CRM)
- Speed to market matters more than perfection
  # 上市速度比完美更重要
- You're still figuring out the process
  # 你还在摸索流程
- Budget is limited                 # 预算有限
- No technical team available       # 没有技术团队

THE PATTERN: Start with existing tools, identify
gaps, build custom only for those gaps.
# 模式:从现有工具开始,识别差距,
# 只为那些差距定制构建。
\end{lstlisting}
\end{codebox}

我犯了过早定制构建的错误。我花了三周时间创建一个定制的邮件分流系统,而Claude Pro和Zapier的组合可以处理百分之九十的工作。那三周本可以用来获取客户。

必须时才构建。能买时就买。你的竞争优势不是你的基础设施——而是你用它做什么。

\section{三层架构}

我开始把智能体基础设施想象成三层,每层在前一层基础上构建。大多数独立创始人应该从第一层开始,只有在超出它时才升级。

\subsection{第一层:无代码(每个人都应该从这里开始)}

这是你验证AI智能体是否真正适用于你的业务的地方。投资最小。学习最大。

\begin{codebox}
\begin{lstlisting}[style=python]
THE NO-CODE STACK                   # 无代码技术栈

Chat Interfaces (Pick one):         # 聊天界面(选一个):
|-- Claude.ai Pro ($20/mo)
|   `-- Best for: Complex reasoning, long documents
|       # 最适合:复杂推理,长文档
|-- ChatGPT Plus ($20/mo)
|   `-- Best for: Custom GPTs, plugin ecosystem
|       # 最适合:自定义GPTs,插件生态
`-- Gemini Advanced ($20/mo)
    `-- Best for: Google Workspace integration
        # 最适合:Google Workspace集成

Workflow Automation (Pick one):     # 工作流自动化(选一个):
|-- Zapier ($20-50/mo)
|   `-- Best for: Simple if-this-then-that
|       # 最适合:简单的如果-那么
|-- Make ($9-16/mo)
|   `-- Best for: Complex multi-step workflows
|       # 最适合:复杂的多步骤工作流
`-- n8n ($0 self-hosted, $20/mo cloud)
    `-- Best for: AI-native, developer-friendly
        # 最适合:AI原生,开发者友好

AI-Native Tools (Add as needed):    # AI原生工具(按需添加):
|-- Lindy ($49/mo)
|   `-- Best for: Email, calendar, research
|       # 最适合:邮件、日历、研究
|-- Bardeen ($10/mo)
|   `-- Best for: Browser automation
|       # 最适合:浏览器自动化
`-- Relay.app ($50/mo)
    `-- Best for: Human-in-the-loop workflows
        # 最适合:人在循环中的工作流

TOTAL INVESTMENT: $100-200/month    # 总投资:$100-200/月
CAPABILITY: 80% of what you need    # 能力:你需要的80%
TIME TO DEPLOY: Days, not weeks     # 部署时间:几天,不是几周
\end{lstlisting}
\end{codebox}

以下是这个技术栈在实践中的样子。每天早上,Claude Pro读取我的邮件(通过Make集成),对它们进行分类,起草回复,并将它们放入审核队列。Zapier处理机械工作——转发、归档、安排。我花十五分钟审核,而不是两小时淹没。

\subsection{第二层:低代码(当你超出第一层时)}

当你开始遇到限制时,你就会知道你已经超出第一层:无代码工具无法处理你的边缘情况,多个订阅的成本超过了构建的成本,或者你需要不存在的集成。

\begin{codebox}
\begin{lstlisting}[style=python]
THE LOW-CODE STACK                  # 低代码技术栈

Visual Agent Builders:              # 可视化智能体构建器:
|-- Flowise (open source, free)
|   `-- Visual LangChain builder, self-hosted
|       # 可视化LangChain构建器,自托管
|-- Dify.ai (free tier available)
|   `-- Visual agent workflows, managed
|       # 可视化智能体工作流,托管
`-- Voiceflow ($50/mo)
    `-- Voice and chat agent builder
        # 语音和聊天智能体构建器

Database + AI:                      # 数据库 + AI:
|-- Supabase + pgvector (generous free tier)
|   `-- PostgreSQL with vector search
|       # 带向量搜索的PostgreSQL
|-- Pinecone ($0-70/mo)
|   `-- Purpose-built vector database
|       # 专用向量数据库
`-- Weaviate (open source)
    `-- Self-hosted vector search
        # 自托管向量搜索

Custom Integrations:                # 定制集成:
|-- Pipedream (free-$19/mo)
|   `-- Code-level workflow builder
|       # 代码级工作流构建器
|-- Retool ($10/mo)
|   `-- Internal tools with AI
|       # 带AI的内部工具
`-- Airplane (free tier)
    `-- Task and workflow runner
        # 任务和工作流运行器

TOTAL INVESTMENT: $50-200/month + time
                                    # 总投资:$50-200/月 + 时间
CAPABILITY: 95% of what you need    # 能力:你需要的95%
TIME TO DEPLOY: Weeks               # 部署时间:几周
\end{lstlisting}
\end{codebox}

在这一层,你在构建真正的系统。你的客户知识库存在于向量数据库中。你的智能体查询它以获取上下文。你的工作流处理复杂的分支逻辑。

\subsection{第三层:代码优先(最大控制)}

这是当AI是你产品的核心、当你需要完全控制、或者当你运营在定制优化重要的规模时你要去的地方。

\begin{codebox}
\begin{lstlisting}[style=python]
THE CODE-FIRST STACK                # 代码优先技术栈

AI SDKs:                            # AI SDK:
|-- Anthropic SDK (Python/TypeScript)
|   `-- Direct Claude API access    # 直接Claude API访问
|-- OpenAI SDK (Python/TypeScript)
|   `-- GPT models and assistants   # GPT模型和助手
`-- LangChain / LlamaIndex
    `-- Agent frameworks with tools # 带工具的智能体框架

Infrastructure:                     # 基础设施:
|-- Modal ($0-30/mo)
|   `-- Serverless Python execution # 无服务器Python执行
|-- Railway ($5/mo)
|   `-- Simple app deployment       # 简单应用部署
`-- Fly.io ($0-20/mo)
    `-- Global edge deployment      # 全球边缘部署

Observability:                      # 可观测性:
|-- LangSmith (free tier)
|   `-- LangChain tracing and debugging
|       # LangChain追踪和调试
|-- Helicone (free tier)
|   `-- LLM observability and analytics
|       # LLM可观测性和分析
`-- Weights & Biases
    `-- ML experiment tracking      # ML实验追踪

Orchestration:                      # 编排:
|-- Temporal (open source)
|   `-- Workflow orchestration      # 工作流编排
|-- Prefect (free tier)
|   `-- Data pipeline orchestration # 数据管道编排
`-- Dagster (free tier)
    `-- Asset-based orchestration   # 基于资产的编排

TOTAL INVESTMENT: $50-200/month + development
                                    # 总投资:$50-200/月 + 开发
CAPABILITY: Everything you can imagine
                                    # 能力:你能想象的一切
TIME TO DEPLOY: Months of development
                                    # 部署时间:几个月的开发
\end{lstlisting}
\end{codebox}

大多数独立创始人永远不需要第三层。但如果你需要,你会知道——因为第二层的限制会感觉像手铐。

\section{按业务类型的技术栈}

让我根据常见的商业模式给你具体的建议。

\subsection{电子商务/产品业务}

销售实体产品意味着管理订单、库存、发货和客户服务。以下是有效的技术栈:

\begin{codebox}
\begin{lstlisting}[style=python]
E-COMMERCE STACK                    # 电商技术栈

Core Platform:                      # 核心平台:
|-- Shopify (your store)            # Shopify(你的商店)
|-- Claude Pro (agent brain)        # Claude Pro(智能体大脑)
|-- n8n (orchestration)             # n8n(编排)
`-- Notion (knowledge base)         # Notion(知识库)

Agent Configuration:                # 智能体配置:
|-- Oscar: Order processing         # Oscar:订单处理
|   Flow: Shopify -> n8n -> Claude -> Shipping
|-- Casey: Customer support         # Casey:客户支持
|   Flow: Email -> Claude -> Response/Escalate
|-- Finn: Invoicing and reconciliation
|   # Finn:开票和对账
|   Flow: Orders -> n8n -> Stripe -> Notifications
`-- Maya: Product descriptions      # Maya:产品描述
    Flow: Photos -> Claude -> Shopify listings

Integration Points:                 # 集成点:
|-- Shopify webhooks trigger n8n workflows
|   # Shopify webhooks触发n8n工作流
|-- Email (Gmail API) connects to Claude
|   # 邮件(Gmail API)连接到Claude
|-- Shipping (EasyPost/ShipStation) automated
|   # 发货(EasyPost/ShipStation)自动化
`-- Payments (Stripe) fully automated
    # 支付(Stripe)完全自动化

Monthly Cost: $200-300 total        # 月度成本:总计$200-300
\end{lstlisting}
\end{codebox}

电商的关键洞见:自动化重复性的,升级例外的。百分之九十的订单是直接的。你的智能体处理那些。需要关注的百分之十被路由给你。

\subsection{服务业务/代理商}

服务业务靠关系和交付物运行。技术栈专注于客户沟通和项目管理。

\begin{codebox}
\begin{lstlisting}[style=python]
SERVICE BUSINESS STACK              # 服务业务技术栈

Core Platform:                      # 核心平台:
|-- Cal.com (scheduling)            # Cal.com(排程)
|-- Claude Pro (agent brain)        # Claude Pro(智能体大脑)
|-- Make or n8n (orchestration)     # Make或n8n(编排)
`-- Obsidian (client knowledge)     # Obsidian(客户知识)

Agent Configuration:                # 智能体配置:
|-- Emma: Email and calendar management
|   # Emma:邮件和日历管理
|   Flow: Gmail -> Claude -> Response/Schedule
|-- Sam: Lead qualification and intake
|   # Sam:线索筛选和接入
|   Flow: Form -> Claude -> CRM -> Follow-up
|-- Casey: Client success and check-ins
|   # Casey:客户成功和跟进
|   Flow: Schedule -> Claude -> Outreach
`-- Finn: Proposals and invoicing
    # Finn:提案和开票
    Flow: Scope -> Claude -> Proposal -> Invoice

Integration Points:                 # 集成点:
|-- Cal.com syncs with Google Calendar
|   # Cal.com与Google Calendar同步
|-- Forms (Typeform/Tally) feed into CRM
|   # 表单(Typeform/Tally)输入CRM
|-- Stripe handles payment collection
|   # Stripe处理收款
`-- Slack notifications for urgent items
    # Slack通知紧急事项

Monthly Cost: $150-250 total        # 月度成本:总计$150-250
\end{lstlisting}
\end{codebox}

对于服务业务,关键的自动化是客户沟通之间的空间。每封邮件都得到及时回复。每次跟进都按计划进行。没有东西漏掉。

\subsection{SaaS/软件产品}

如果你在构建软件,你的智能体可以直接与你的产品集成。

\begin{codebox}
\begin{lstlisting}[style=python]
SAAS STACK                          # SaaS技术栈

Core Platform:                      # 核心平台:
|-- Your application (the product)  # 你的应用(产品)
|-- Claude API (agent brain)        # Claude API(智能体大脑)
|-- Supabase (database and auth)    # Supabase(数据库和认证)
`-- Vercel (deployment)             # Vercel(部署)

Agent Configuration:                # 智能体配置:
|-- Sam: Trial to paid conversion   # Sam:试用转付费
|   Flow: Usage signals -> Outreach
|-- Casey: Onboarding and support   # Casey:引导和支持
|   Flow: Intercom -> Claude -> Response
|-- Maya: Content and SEO           # Maya:内容和SEO
|   Flow: Claude -> Blog -> Social
`-- Oscar: DevOps and monitoring    # Oscar:DevOps和监控
    Flow: Alerts -> Triage -> Escalate

Integration Points:                 # 集成点:
|-- Product events stream to Supabase
|   # 产品事件流向Supabase
|-- Supabase triggers Claude analysis
|   # Supabase触发Claude分析
|-- Intercom handles support automation
|   # Intercom处理支持自动化
`-- GitHub connects to deployment
    # GitHub连接到部署

Monthly Cost: $100-200 (plus API usage)
                                    # 月度成本:$100-200(加API使用)
\end{lstlisting}
\end{codebox}

SaaS业务有一个优势:你的用户已经在你的系统中。你知道他们的行为。你的智能体可以根据使用模式主动出击,而不是对邮件被动反应。

\section{核心集成模式}

无论你选择哪个技术栈,你的智能体都会遵循几个基本模式。掌握这些,你几乎可以构建任何东西。

\subsection{模式1:事件 $\rightarrow$ AI $\rightarrow$ 行动}

这是最常见的模式。某事发生,AI处理它,采取行动。

\begin{codebox}
\begin{lstlisting}[style=python]
TRIGGER -> PROCESS -> ACT           # 触发 -> 处理 -> 行动

Example: New Lead Processing        # 示例:新线索处理

1. TRIGGER                          # 1. 触发
   Typeform webhook fires with new submission
   # Typeform webhook触发新提交
   {name: "Sarah", email: "sarah@techcorp.com",
    company: "TechCorp", message: "Need help with..."}

2. PROCESS (n8n + Claude)           # 2. 处理(n8n + Claude)
   - Enrich with LinkedIn data      # 用LinkedIn数据丰富
   - Score against BANT criteria    # 根据BANT标准评分
   - Result: {score: 85, tier: "hot", summary: "..."}
     # 结果:{分数:85,级别:"热",摘要:"..."}

3. ACT                              # 3. 行动
   - Create CRM record              # 创建CRM记录
   - Send personalized response     # 发送个性化回复
   - Schedule follow-up task        # 安排跟进任务
   - Notify you via Slack           # 通过Slack通知你

Time: Under 60 seconds from submission to response
# 时间:从提交到响应不到60秒
\end{lstlisting}
\end{codebox}

\subsection{模式2:计划 $\rightarrow$ 收集 $\rightarrow$ 综合}

这个模式支持你所有的报告和监控。按计划,收集数据,让AI综合洞察。

\begin{codebox}
\begin{lstlisting}[style=python]
SCHEDULE -> GATHER -> SYNTHESIZE    # 计划 -> 收集 -> 综合

Example: Daily Business Summary     # 示例:每日业务摘要

1. SCHEDULE                         # 1. 计划
   6 AM daily trigger               # 每天早上6点触发

2. GATHER                           # 2. 收集
   - Pull yesterday's sales from Stripe
     # 从Stripe拉取昨天的销售
   - Check support ticket status    # 检查支持工单状态
   - Get marketing metrics from analytics
     # 从分析获取营销指标
   - Review cash position           # 审查现金状况

3. SYNTHESIZE (Claude)              # 3. 综合(Claude)
   Generate executive summary with: # 生成执行摘要包含:
   - Key metrics and comparisons    # 关键指标和对比
   - Alerts and concerns            # 警报和关注点
   - Recommendations for today      # 今天的建议

4. DELIVER                          # 4. 交付
   Email summary to your inbox before you wake up
   # 在你醒来前将摘要发到你的收件箱
\end{lstlisting}
\end{codebox}

\subsection{模式3:监控 $\rightarrow$ 检测 $\rightarrow$ 响应}

这个模式监视问题并在它们成为危机之前响应。

\begin{codebox}
\begin{lstlisting}[style=python]
WATCH -> IDENTIFY -> ACT            # 监视 -> 识别 -> 行动

Example: Customer Health Monitoring # 示例:客户健康监控

1. WATCH                            # 1. 监视
   Every 4 hours, pull usage metrics:
   # 每4小时,拉取使用指标:
   - Login frequency                # 登录频率
   - Feature usage                  # 功能使用
   - Support ticket patterns        # 支持工单模式

2. IDENTIFY (Claude)                # 2. 识别(Claude)
   Analyze for warning signs:       # 分析警告信号:
   "Customer usage dropped 50% in 7 days"
   # "客户使用量在7天内下降50%"
   "No login in 14 days"            # "14天未登录"
   "Negative sentiment in recent ticket"
   # "最近工单中的负面情绪"

3. ACT                              # 3. 行动
   - Send proactive check-in email  # 发送主动跟进邮件
   - Create internal alert          # 创建内部警报
   - If no response in 48h, escalate # 如48h无回复,升级
   - If health critical, notify founder immediately
     # 如健康危急,立即通知创始人
\end{lstlisting}
\end{codebox}

\subsection{模式4:请求 $\rightarrow$ 丰富 $\rightarrow$ 完成}

这个模式支持内容创作和研究任务。获取输入,用上下文丰富,产生输出。

\begin{codebox}
\begin{lstlisting}[style=python]
INPUT -> AUGMENT -> OUTPUT          # 输入 -> 增强 -> 输出

Example: Content Creation           # 示例:内容创作

1. INPUT                            # 1. 输入
   "Write a blog post about customer churn"
   # "写一篇关于客户流失的博客帖子"

2. AUGMENT                          # 2. 增强
   - Search knowledge base for relevant notes
     # 搜索知识库相关笔记
   - Pull recent customer stories and examples
     # 拉取最近的客户故事和示例
   - Get current statistics from industry sources
     # 从行业来源获取当前统计数据
   - Load brand voice guide         # 加载品牌语调指南

3. COMPLETE (Claude with context)   # 3. 完成(Claude带上下文)
   Generate 1500-word blog post that:
   # 生成1500字的博客帖子:
   - Matches your voice             # 匹配你的语调
   - Includes real examples         # 包含真实示例
   - Cites accurate data            # 引用准确数据

4. OUTPUT                           # 4. 输出
   - Save draft to content folder   # 保存草稿到内容文件夹
   - Generate 3 social snippets     # 生成3条社交片段
   - Queue for human review         # 排队等待人工审核
\end{lstlisting}
\end{codebox}

\section{为什么我推荐n8n}

在尝试了Zapier、Make、Pipedream和定制代码之后,我选定n8n作为我的编排层。原因如下:

\begin{codebox}
\begin{lstlisting}[style=python]
N8N ADVANTAGES                      # N8N优势

Open Source:                        # 开源:
- Self-host for free (or $20/mo cloud)
  # 免费自托管(或$20/月云端)
- No vendor lock-in                 # 无供应商锁定
- Full control over your data       # 完全控制你的数据
- Inspect and modify anything       # 检查和修改任何东西

AI-Native:                          # AI原生:
- Built-in nodes for Claude, OpenAI, etc.
  # 内置Claude、OpenAI等节点
- Vector database support           # 向量数据库支持
- Code nodes for custom logic       # 代码节点用于定制逻辑
- Designed for AI workflows         # 为AI工作流设计

Integration Rich:                   # 丰富的集成:
- 400+ pre-built integrations       # 400+预构建集成
- HTTP node for anything else       # HTTP节点用于其他
- Webhooks in and out               # 入站和出站Webhooks
- Direct database connections       # 直接数据库连接

Developer Friendly:                 # 开发者友好:
- JavaScript/Python in code nodes   # 代码节点中的JavaScript/Python
- Git-based version control possible # 可以基于Git版本控制
- Environment variables             # 环境变量
- Self-documenting workflows        # 自文档化工作流
\end{lstlisting}
\end{codebox}

让我向你展示一个真实的工作流。这是我如何处理线索:

\begin{codebox}
\begin{lstlisting}[style=python]
N8N WORKFLOW: Lead Processor        # N8N工作流:线索处理器

Node 1: [Webhook]                   # 节点1:[Webhook]
- Receives POST /webhook/new-lead   # 接收POST /webhook/new-lead
- Body: {name, email, company, message}

Node 2: [HTTP Request]              # 节点2:[HTTP请求]
- Enriches with company data from Clearbit
  # 用Clearbit公司数据丰富
- Adds: {size, industry, funding}   # 添加:{规模,行业,融资}

Node 3: [Claude]                    # 节点3:[Claude]
- System: "You are a lead qualification expert..."
  # 系统:"你是一个线索筛选专家..."
- Prompt: "Score this lead 0-100 based on..."
  # 提示:"根据...将此线索评分0-100"
- Output: {score: 85, tier: "hot", summary: "..."}
  # 输出:{分数:85,级别:"热",摘要:"..."}

Node 4: [IF]                        # 节点4:[IF]
- If score >= 80: Hot path          # 如果分数>=80:热路径
- If score >= 50: Warm path         # 如果分数>=50:温路径
- Else: Cold path                   # 否则:冷路径

Node 5a (Hot): [Parallel]           # 节点5a(热):[并行]
- Create Notion record              # 创建Notion记录
- Send immediate response           # 发送即时响应
- Slack notification                # Slack通知
- Schedule follow-up                # 安排跟进

Node 5b (Warm): [Parallel]          # 节点5b(温):[并行]
- Create Notion record              # 创建Notion记录
- Add to nurture sequence           # 添加到培育序列
- Send welcome email                # 发送欢迎邮件

Node 5c (Cold): [Single]            # 节点5c(冷):[单一]
- Log and archive                   # 记录和归档

Total nodes: 8                      # 总节点:8
Execution time: ~3 seconds          # 执行时间:约3秒
Cost: ~$0.01 per lead               # 成本:约每条线索$0.01
\end{lstlisting}
\end{codebox}

\section{管理API成本}

如果你不谨慎,AI API成本可能会快速攀升。以下是我如何控制它们。

\subsection{理解代币经济学}

\begin{codebox}
\begin{lstlisting}[style=python]
CLAUDE API PRICING (as of 2026)     # Claude API定价(截至2026年)

Model                    Input/1M    Output/1M
# 模型                    输入/1M     输出/1M
------------------------  --------   ---------
Claude 3.5 Sonnet        $3.00       $15.00
Claude 3.5 Haiku         $0.25       $1.25
Claude 3 Opus            $15.00      $75.00

TYPICAL COST PER TASK               # 每任务典型成本

Email triage (Haiku):               # 邮件分流(Haiku):
- Input: ~500 tokens = $0.000125    # 输入:约500代币
- Output: ~100 tokens = $0.000125   # 输出:约100代币
- Total: ~$0.00025 per email        # 总计:每封邮件约$0.00025

Lead qualification (Sonnet):        # 线索筛选(Sonnet):
- Input: ~1,000 tokens = $0.003     # 输入:约1,000代币
- Output: ~300 tokens = $0.0045     # 输出:约300代币
- Total: ~$0.008 per lead           # 总计:每条线索约$0.008

Blog post (Sonnet):                 # 博客帖子(Sonnet):
- Input: ~2,000 tokens = $0.006     # 输入:约2,000代币
- Output: ~2,000 tokens = $0.03     # 输出:约2,000代币
- Total: ~$0.04 per post            # 总计:每篇帖子约$0.04
\end{lstlisting}
\end{codebox}

\subsection{成本优化策略}

\begin{codebox}
\begin{lstlisting}[style=python]
REDUCE COSTS BY 70%+                # 降低成本70%+

1. USE HAIKU FOR SIMPLE TASKS       # 1. 简单任务用HAIKU
   - Classification: Haiku          # 分类:Haiku
   - Entity extraction: Haiku       # 实体提取:Haiku
   - Simple Q&A: Haiku              # 简单问答:Haiku
   - Sonnet only for reasoning tasks # 只用Sonnet做推理任务
   - Savings: 90% vs using Sonnet for everything
     # 节省:比所有都用Sonnet节省90%

2. CACHE REPEATED QUERIES           # 2. 缓存重复查询
   - Same playbook = cached response # 相同策略手册=缓存响应
   - FAQ answers = cached           # FAQ答案=缓存
   - Simple Redis or file-based cache
     # 简单Redis或基于文件的缓存
   - Savings: 50%+ on repeated operations
     # 节省:重复操作50%+

3. BATCH OPERATIONS                 # 3. 批量操作
   - Process emails in batches of 10 # 10封一批处理邮件
   - Daily summaries, not hourly    # 每日摘要,不是每小时
   - Fewer API calls = lower costs  # 更少API调用=更低成本
   - Savings: 30% on overhead       # 节省:开销30%

4. TRUNCATE CONTEXT                 # 4. 截断上下文
   - Only send relevant history     # 只发送相关历史
   - Summarize long documents       # 总结长文档
   - Don't include entire customer file
     # 不要包含整个客户文件
   - Savings: 40% on token usage    # 节省:代币使用40%

5. PROMPT EFFICIENCY                # 5. 提示效率
   - Clear, concise instructions    # 清晰简洁的指令
   - Structured output (JSON)       # 结构化输出(JSON)
   - Don't ask for explanations you won't use
     # 不要要求你不会用的解释
   - Savings: 20% on verbosity      # 节省:冗长度20%
\end{lstlisting}
\end{codebox}

\subsection{月度预算示例}

\begin{codebox}
\begin{lstlisting}[style=python]
REALISTIC COST PROJECTIONS          # 现实成本预测

Small Business (50 leads/mo, 100 emails/day):
# 小型业务(50线索/月,100邮件/天):
- Email processing: 3,000/mo x $0.00025 = $0.75
  # 邮件处理
- Lead qualification: 50/mo x $0.008 = $0.40
  # 线索筛选
- Content (4 posts): 4 x $0.04 = $0.16
  # 内容(4篇帖子)
- Daily summaries: 30 x $0.03 = $0.90
  # 每日摘要
- Support tickets: 50 x $0.01 = $0.50
  # 支持工单
- TOTAL: ~$3/month in API costs     # 总计:API成本约$3/月
- Add buffer: ~$10-15/month realistic
  # 加缓冲:现实$10-15/月

Medium Business (200 leads/mo, 300 emails/day):
# 中型业务(200线索/月,300邮件/天):
- Email processing: 9,000/mo x $0.00025 = $2.25
- Lead qualification: 200/mo x $0.008 = $1.60
- Content (12 posts): 12 x $0.04 = $0.48
- Daily summaries: 30 x $0.03 = $0.90
- Support tickets: 200 x $0.01 = $2.00
- TOTAL: ~$7/month in API costs     # 总计:API成本约$7/月
- Add buffer: ~$25-35/month realistic
  # 加缓冲:现实$25-35/月

Growing Business (500 leads/mo, 500 emails/day):
# 成长中业务(500线索/月,500邮件/天):
- TOTAL: ~$15/month in API costs    # 总计:API成本约$15/月
- Add buffer: ~$50-75/month realistic
  # 加缓冲:现实$50-75/月

These costs are remarkably low.
The infrastructure overhead costs more than the AI.
# 这些成本非常低。
# 基础设施开销比AI成本更高。
\end{lstlisting}
\end{codebox}

\section{安全和数据隐私}

这是许多创始人紧张的地方——而且有充分的理由。你在向AI提供商发送客户数据。以下是如何安全地做到这一点。

\begin{codebox}
\begin{lstlisting}[style=python]
DATA SECURITY PRINCIPLES            # 数据安全原则

1. MINIMIZE DATA SENT TO AI         # 1. 最小化发送给AI的数据
   - Only what's needed for the task # 只有任务需要的
   - Never passwords or API keys    # 永不发送密码或API密钥
   - Anonymize when possible        # 可能时匿名化
   - Reference IDs, not full records # 引用ID,不是完整记录

2. USE ENTERPRISE TIERS             # 2. 使用企业级别
   - No training on your data (verify in ToS)
     # 不在你的数据上训练(在服务条款中验证)
   - SOC 2 compliance               # SOC 2合规
   - Data retention controls        # 数据保留控制
   - Audit logs available           # 审计日志可用

3. KEEP SENSITIVE DATA LOCAL        # 3. 敏感数据保持本地
   - Full PII stays in your database # 完整PII留在你的数据库
   - Financial details stay in your systems
     # 财务详情留在你的系统
   - Send only necessary context    # 只发送必要上下文
   - "Customer #1234" not "John Smith SSN 123-45-6789"
     # "客户#1234"而不是"John Smith SSN 123-45-6789"

4. AUDIT REGULARLY                  # 4. 定期审计
   - Review what's being sent monthly
     # 每月审查发送了什么
   - Check API logs for anomalies   # 检查API日志异常
   - Update playbooks as needed     # 按需更新策略手册
   - Train any team members on protocols
     # 培训所有团队成员协议
\end{lstlisting}
\end{codebox}

\subsection{合规检查清单}

在你部署处理客户数据的智能体之前:

\begin{codebox}
\begin{lstlisting}[style=python]
PRIVACY COMPLIANCE                  # 隐私合规

[ ] Data processing agreement with AI provider
    # 与AI提供商的数据处理协议
[ ] Privacy policy updated for AI usage
    # 隐私政策已更新以涵盖AI使用
[ ] Customer notification if required (GDPR)
    # 如需要则通知客户(GDPR)
[ ] Opt-out mechanism for AI processing
    # AI处理的退出机制
[ ] Data retention limits configured
    # 数据保留限制已配置
[ ] Audit trail for AI-generated decisions
    # AI生成决策的审计跟踪
[ ] Human review process for critical decisions
    # 关键决策的人工审核流程
[ ] Incident response plan for AI errors
    # AI错误的事件响应计划
\end{lstlisting}
\end{codebox}

\section{你的第一周}

以下是如何开始的,逐日进行。

\textbf{第1-2天:设置账户}

\begin{codebox}
\begin{lstlisting}[style=python]
[ ] Claude Pro subscription ($20/mo)
    # Claude Pro订阅($20/月)
[ ] n8n Cloud account ($20/mo) or self-host
    # n8n Cloud账户($20/月)或自托管
[ ] Notion or Obsidian for knowledge base
    # Notion或Obsidian用于知识库
[ ] Verify all accounts work
    # 验证所有账户工作
\end{lstlisting}
\end{codebox}

\textbf{第3-4天:你的第一个工作流}

\begin{codebox}
\begin{lstlisting}[style=python]
[ ] Connect email (Gmail API)       # 连接邮件(Gmail API)
[ ] Create simple: Email -> Claude -> Draft Response
    # 创建简单的:邮件 -> Claude -> 草稿回复
[ ] Test with your own emails       # 用你自己的邮件测试
[ ] Refine the prompt based on results
    # 根据结果优化提示
\end{lstlisting}
\end{codebox}

\textbf{第5-7天:生产就绪}

\begin{codebox}
\begin{lstlisting}[style=python]
[ ] Add error handling              # 添加错误处理
[ ] Test edge cases                 # 测试边缘情况
[ ] Document the workflow           # 文档化工作流
[ ] Set up basic monitoring         # 设置基本监控
[ ] Go live with supervision        # 带监督上线
\end{lstlisting}
\end{codebox}

\textbf{第2周:第一个真正的智能体}

根据你最大的痛点选择:
\begin{itemize}
\item 被邮件淹没?构建Emma
\item 丢失线索?构建Sam
\item 内容不一致?构建Maya
\item 支持耗时太长?构建Casey
\end{itemize}

\begin{keyinsight}[技术栈公式]
你的智能体基础设施应该是无聊的。可靠的。简单的。

\textbf{从简单开始:}Claude Pro + n8n + Notion = \$50-100/月,有百分之八十的能力。

\textbf{需要时才扩展:}只有当简单性失败时才添加复杂性。

\textbf{核心模式:}每个工作流都是事件 $\rightarrow$ AI $\rightarrow$ 行动,只是触发器和输出不同。

目标不是最复杂的技术栈。目标是解决你问题的最简单的技术栈。还记得我凌晨2点的基础设施崩溃吗?那发生是因为我为能力而不是可靠性优化。

构建无聊的。睡得好。
\end{keyinsight}

\textbf{下一章:}编写让你的智能体一致和可靠的提示、策略手册和协议。
