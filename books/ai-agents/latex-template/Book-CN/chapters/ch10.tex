\chapter{文件夹结构即操作程序}

\section{47,000美元的发票改变了一切}

那封邮件在一个周一早上到达:我们的年度企业软件续约。数据库托管、CRM许可证、项目管理工具、报表仪表板、文档存储。我已经付这些账单好几年了,几乎不看。这次,我看了。

\textbf{47,000美元。每年。对于一个一人公司。}

我开始分解它:

\begin{itemize}
\item Oracle数据库托管:\$8,400
\item Salesforce许可证(三个席位``为了增长''但从未发生):\$5,400
\item Jira和Confluence:\$3,600
\item Tableau,用于我可能一个月运行两次的报表:\$1,680
\item SharePoint和Microsoft 365:\$1,800
\item AWS基础设施:\$12,000
\item Zapier和集成工具:\$2,400
\item 各种SaaS订阅:\$11,000+
\end{itemize}

但钱甚至不是最糟糕的部分。是复杂性。

我每个月大约花八个小时只是维护这些系统——更新权限、修复坏掉的集成、在互不通信的工具之间迁移数据、在供应商决定``改进''他们的产品时学习新界面、在单点登录不可避免地崩溃时与认证搏斗。

我在为一人公司运行企业级基础设施。这种荒谬从未真正击中我,直到我看到那张发票。

\section{意外的发现}

启示来自一个不太可能的来源:一个经营成功独立软件公司的开发者朋友。

我一直在抱怨我的Salesforce集成那个月第三次崩溃了。他问了一个奇怪的问题:``你为什么不直接用markdown文件?''

我笑了。Markdown文件?用来经营业务?那是用来写文档和博客帖子的,不是客户数据和销售管道。

他分享了他的屏幕。他整个业务都在他电脑上的一个文件夹结构中运行。客户文件。项目追踪。财务记录。一切都是纯文本,组织在文件夹中,用Git进行版本控制。

``但你怎么查询?你怎么生成报表?''

``AI读它,''他简单地说。``我问Claude关于我业务的问题,它读取文件并回答。不需要SQL。不需要配置仪表板。''

我持怀疑态度。这肯定无法扩展。肯定有边缘情况。企业工具存在肯定是有原因的。

然后他给我看了他的月度成本:\$187。Obsidian同步、GitHub和Claude API使用。

我回家后无法停止思考它。

\section{企业软件的肮脏秘密}

以下是没人告诉你的关于企业工具的事:

它们是为企业构建的。有数百名员工的公司。数据库管理员团队。专门的IT人员。采购部门通过谈判批量折扣来证明自己存在的价值。

对于独立创始人或小团队,这些工具就像租一艘游轮过河。是的,技术上可行。但你为永远不会使用的能力付费,维护你不需要的复杂性,把时间花在当系统管理员而不是经营业务上。

传统技术栈看起来像这样:

\begin{codebox}
\begin{lstlisting}[style=python]
THE ENTERPRISE TRAP                 # 企业陷阱

Database:    Oracle, SQL Server, PostgreSQL
             # 数据库:Oracle, SQL Server, PostgreSQL
             (Because you "might need" relational queries)
             # (因为你"可能需要"关系查询)

CRM:         Salesforce, Dynamics
             # CRM:Salesforce, Dynamics
             (Because that's what "real businesses" use)
             # (因为那是"真正的企业"使用的)

ERP:         SAP, Oracle, NetSuite
             # ERP:SAP, Oracle, NetSuite
             (Because accountants asked for it once)
             # (因为会计曾经要求过)

Reporting:   Tableau, Power BI, Looker
             # 报表:Tableau, Power BI, Looker
             (Because dashboards look professional)
             # (因为仪表板看起来专业)

Projects:    Jira, Monday, Asana
             # 项目:Jira, Monday, Asana
             (Because agile requires software, apparently)
             # (因为敏捷显然需要软件)

Documents:   SharePoint, Box, Google Drive
             # 文档:SharePoint, Box, Google Drive
             (Because files need to live... somewhere)
             # (因为文件需要存放...某个地方)

Email:       Outlook, Gmail
             # 邮件:Outlook, Gmail
             (This one is actually necessary)
             # (这个实际上是必要的)

Integration: MuleSoft, Boomi, custom ETL
             # 集成:MuleSoft, Boomi, 自定义ETL
             (Because none of the above talk to each other)
             # (因为上面没有一个能互相通信)

TOTAL COST: $10,000-100,000+/month  # 总成本:$10,000-100,000+/月
TEAM REQUIRED: 3-20 people to manage # 所需团队:3-20人管理
IMPLEMENTATION: 6-18 months          # 实施:6-18个月
ACTUAL VALUE: Maybe 10% of features used
             # 实际价值:可能只用了10%的功能
\end{lstlisting}
\end{codebox}

肮脏的秘密:百分之九十的企业不需要这种复杂性。你不需要Oracle。你不需要跨十七个表的关系数据库联接。你不需要商业智能平台。

你需要能读纯文本的AI智能体。

\section{新范式}

在那次对话之后,我做了一个实验。我把一切从企业工具导出到markdown文件。客户变成了带有上下文文件的文件夹。项目变成了检查清单。财务记录变成了结构化文本。所有在数据库和仪表板中的东西现在都在我电脑上的一个文件夹中。

然后我让我的AI智能体指向它。

结果是立即且深刻的:

\begin{codebox}
\begin{lstlisting}[style=python]
THE LIBERATION                       # 解放

Knowledge:   Obsidian, Notion, markdown files
             # 知识:Obsidian, Notion, markdown文件
             (Human-readable AND AI-readable)
             # (人类可读且AI可读)

Publishing:  Git, mdBook, Astro
             # 发布:Git, mdBook, Astro
             (Version-controlled, deployable anywhere)
             # (版本控制,可部署到任何地方)

Projects:    GitHub Issues, Linear, plain text
             # 项目:GitHub Issues, Linear, 纯文本
             (Checkboxes work surprisingly well)
             # (复选框出奇地好用)

Data:        Markdown tables, YAML, JSON
             # 数据:Markdown表格, YAML, JSON
             (Structured enough, flexible enough)
             # (足够结构化,足够灵活)

Reporting:   AI-generated from context
             # 报表:从上下文AI生成
             (Ask any question, get any answer)
             # (问任何问题,得到任何答案)

Integration: MCP, APIs, AI agents
             # 集成:MCP, APIs, AI智能体
             (Agents read files, no ETL needed)
             # (智能体读文件,不需要ETL)

Documents:   Markdown with AI assistance
             # 文档:Markdown配合AI辅助
             (Write once, use everywhere)
             # (一次编写,到处使用)

Email:       AI agents read/write
             # 邮件:AI智能体读写
             (Handled, not managed)
             # (被处理,而不是被管理)

TOTAL COST: $200-500/month           # 总成本:$200-500/月
TEAM: You + AI agents                # 团队:你 + AI智能体
IMPLEMENTATION: Days, not months     # 实施:几天,而不是几个月
FLEXIBILITY: Infinite                # 灵活性:无限
\end{lstlisting}
\end{codebox}

\section{为什么纯文本胜出}

我逐渐理解为什么这种方式如此有效。纯文本具有企业软件花了几十年试图复制但做得很差的特性。

人类可读性。打开一个markdown文件,你就知道里面有什么。不需要培训。不需要学习界面。不需要理解模式。客户文件读起来像文档,因为它就是文档。

AI可读性。这是游戏规则改变者。AI模型理解自然语言。当你的业务数据存储为自然语言时,AI可以原生读取它。不需要连接器。不需要API。不需要数据转换。AI只是...读取你的文件。

版本控制。Git永远追踪每个文件的每一个变化。谁改了什么,什么时候,为什么。回滚到历史上的任何一点。分支和合并。协作而不冲突。全部免费,使用开发者已经用了几十年的工具。

可移植性。Markdown文件到处都能用。文本编辑器、笔记应用、网页浏览器、手机、AI助手。你永远不会被锁定。随时切换工具。你的数据始终是你的。

速度。纯文本是即时的。不需要查询优化。不需要索引维护。不需要连接池。搜索就是grep。AI在毫秒内读取文件。

\begin{table}[H]
\centering
\small
\begin{tabular}{@{}lll@{}}
\toprule
\textbf{能力} & \textbf{企业技术栈} & \textbf{Markdown技术栈} \\
\midrule
人类可读 & 需要培训 & 任何人都能读 \\
AI可读 & 需要定制集成 & 原生 \\
版本控制 & 复杂(如果有的话) & Git \\
搜索 & 需要授权工具 & grep \\
可移植性 & 供应商锁定 & 通用 \\
速度 & 慢(查询) & 即时 \\
成本 & 数千美元/月 & 几乎免费 \\
\bottomrule
\end{tabular}
\end{table}

\section{替代数据库}

这是我最大的心理障碍。你肯定需要一个正式的数据库吧?你肯定不能用文本文件来经营业务吧?

让我向你展示区别。

\textbf{旧方式}——从传统数据库获取客户上下文:

\begin{codebox}
\begin{lstlisting}[style=python]
-- 7 tables, 15 joins to understand one customer
-- 7个表,15个联接才能理解一个客户

SELECT c.*, a.*, o.*, t.*, n.*, p.*, s.*
FROM customers c
LEFT JOIN addresses a ON c.id = a.customer_id
LEFT JOIN orders o ON c.id = o.customer_id
LEFT JOIN tickets t ON c.id = t.customer_id
LEFT JOIN notes n ON c.id = n.customer_id
LEFT JOIN payments p ON c.id = p.customer_id
LEFT JOIN subscriptions s ON c.id = s.customer_id
WHERE c.id = 12345;

-- Result: JSON blob that needs parsing
-- 结果:需要解析的JSON块
-- Context: Lost across normalized tables
-- 上下文:丢失在规范化表中
-- AI readable: Only with custom integration
-- AI可读:仅通过定制集成
-- Time to understand: Minutes of analysis
-- 理解时间:几分钟的分析
\end{lstlisting}
\end{codebox}

\textbf{新方式}——markdown文件中的客户上下文:

\begin{codebox}
\begin{lstlisting}[style=bash]
# John Smith - Acme Corp

## Context
## 背景
VP of Engineering at Acme Inc. Referred by Sarah Chen.
Active customer since March 2024. Champion user.
# Acme Inc.工程副总裁。由Sarah Chen推荐。
# 2024年3月起的活跃客户。支持者用户。

## History
## 历史

### Deal (March 2024)
### 交易(2024年3月)
- Closed $24,000/year Pro plan    # 成交$24,000/年专业版
- Decision made quickly after demo # 演示后快速决策
- Key feature that sold him: API integrations
  # 打动他的关键功能:API集成

### Support (2024)
### 支持(2024)
- 3 tickets, all resolved same day # 3个工单,全部当天解决
- CSAT: 5/5 on all three          # 满意度:全部5/5
- Never escalated                  # 从未升级

### Expansion (October 2024)
### 扩展(2024年10月)
- Added 5 seats ($500/mo additional) # 增加5个席位
- Requested enterprise features    # 请求企业功能
- Budget approved for Q1           # Q1预算已批准

## Notes
## 备注
- Prefers async communication (email over calls)
  # 偏好异步沟通(邮件优于电话)
- Don't call before 10 AM Pacific  # 太平洋时间上午10点前不要打电话
- Allergic to sales tactics        # 讨厌销售技巧
- Will refer other customers if asked nicely
  # 如果好好问会推荐其他客户

## Linked
## 关联
- [[Orders/12345]]
- [[Support/Ticket-789]]
- [[Calls/2024-03-15]]
\end{lstlisting}
\end{codebox}

AI读取这个文件就能理解一切。客户的性格。他们的沟通偏好。他们与你的历史。他们的扩展潜力。他们推荐的可能性。

没有联接。没有模式。没有ORM。没有数据库管理。只有上下文,人类和AI都可读。

\section{替代报表工具}

这个转变最让我惊讶。我以为我需要Tableau或Power BI来生成报表。仪表板感觉很专业。图表感觉很重要。

然后我发现了一件让人解放的事:AI从原始数据生成的报表比任何BI工具都好。

\textbf{旧方式:}

\begin{enumerate}
\item 登录Tableau
\item 导航到销售仪表板
\item 设置日期过滤器为上周
\item 等待查询执行
\item 盯着图表,试图提取含义
\item 导出为PDF
\item 发邮件给利益相关者
\item 总时间:十五到二十分钟
\item 获得的洞察:可能一两个
\end{enumerate}

\textbf{新方式:}

\begin{codebox}
\begin{lstlisting}[style=python]
You: "How did sales do last week compared to the
previous week? What should I know?"
# 你:"上周销售和前一周相比怎么样?我应该知道什么?"

AI Agent (reading markdown sales files):
# AI智能体(读取markdown销售文件):

Last Week Performance:              # 上周表现:
---------------------
Revenue: $28,500 (+12% week-over-week)
                                    # 收入:$28,500(周环比+12%)
Deals closed: 8 (+3 from prior week)
                                    # 成交:8单(比前一周+3)
Average deal size: $3,562 (+$423)   # 平均单价:$3,562(+$423)
Win rate: 32% (+5 percentage points)
                                    # 赢单率:32%(+5个百分点)

Notable Developments:               # 值得注意的发展:
- Largest deal: TechCorp at $8,500  # 最大单:TechCorp $8,500
- Fastest close: Martinez Design in 3 days
                                    # 最快成交:Martinez Design 3天
- Lost deals: 2 to Competitor X (pricing cited)
                                    # 丢单:2单给竞争对手X(提到价格)

By Tier:                            # 按级别:
1. Enterprise: 4 deals, $22,000 (strong)
                                    # 企业版:4单,$22,000(强劲)
2. Pro: 3 deals, $5,500 (steady)    # 专业版:3单,$5,500(稳定)
3. Starter: 1 deal, $1,000 (needs attention)
                                    # 入门版:1单,$1,000(需关注)

Recommendation: Enterprise pipeline is strong.
Consider pushing Pro -> Enterprise upgrades.
TechCorp might refer - they mentioned loving the
product in their closing call.
# 建议:企业版管道强劲。考虑推动专业版->企业版升级。
# TechCorp可能会推荐——他们在成交电话中提到很喜欢产品。

Risk: Two losses to Competitor X on price. May
need to revisit pricing for competitive situations.
# 风险:两单因价格输给竞争对手X。可能需要
# 重新审视竞争情况下的定价。

Time: 3 seconds                     # 时间:3秒
Insights gained: Actionable, specific, contextual
                                    # 获得的洞察:可行动、具体、有上下文
\end{lstlisting}
\end{codebox}

AI不只是给我看数字。它告诉我什么重要。它给出建议。它连接我可能错过的点。它读取实际的笔记和上下文,而不只是聚合的指标。

这才是报表本应有的样子。

\section{替代项目管理}

我曾经花在更新Jira上的时间比做实际工作还多。创建工单。链接史诗。更新状态。梳理待办事项。参加站会讨论Jira里有什么。

项目管理软件已经变成了关于工作的工作。

Markdown替代方案简单得几乎令人尴尬:

\begin{codebox}
\begin{lstlisting}[style=python]
/projects
|-- active/                         # 活跃/
|   |-- website-redesign.md         # 网站重设计.md
|   |-- mobile-app.md               # 移动应用.md
|   `-- email-migration.md          # 邮件迁移.md
|-- completed/                      # 已完成/
|   `-- q4-campaign.md              # Q4活动.md
`-- ideas/                          # 想法/
    `-- future-features.md          # 未来功能.md
\end{lstlisting}
\end{codebox}

每个项目是一个文件:

\begin{codebox}
\begin{lstlisting}[style=bash]
# Website Redesign
# 网站重设计

## Status: In Progress
## 状态:进行中
**Owner:** Me                       # 负责人:我
**Started:** 2026-01-15             # 开始日期:2026-01-15
**Target:** 2026-02-15              # 目标日期:2026-02-15

## Why We're Doing This
## 为什么要做这个
Current site is slow (3.2s load time) and converts
poorly (1.2% signup rate). Customers have complained.
We're losing deals because the site looks dated.
# 当前网站慢(3.2秒加载时间)且转化差(1.2%注册率)。
# 客户抱怨过。我们因为网站看起来过时而丢单。

## Success Looks Like
## 成功的样子
- Load time under 1 second          # 加载时间在1秒以内
- Signup rate above 3%              # 注册率超过3%
- Mobile-first design               # 移动优先设计
- Blog integrated                   # 博客集成
- Customers stop mentioning it in sales calls
  # 客户不再在销售电话中提到它

## Current Focus
## 当前重点
Week of Jan 27: Homepage complete, moving to product pages
# 1月27日那周:首页完成,进入产品页

## Tasks
## 任务
- [x] Design system in v0           # 在v0中的设计系统
- [x] Homepage development          # 首页开发
- [ ] Product page (in progress)    # 产品页(进行中)
- [ ] About page                    # 关于页
- [ ] Contact + forms               # 联系+表单
- [ ] Blog integration              # 博客集成
- [ ] Testing                       # 测试
- [ ] Launch                        # 上线

## Blockers
## 阻碍
None currently. Last blocker was hosting decision,
resolved Jan 22 (going with Vercel).
# 当前没有。上一个阻碍是托管决策,1月22日解决(选择Vercel)。

## Notes
## 笔记
- 2026-01-20: v0 design approved by stakeholders
  # 2026-01-20:v0设计被利益相关者批准
- 2026-01-22: Homepage shipped to staging
  # 2026-01-22:首页已部署到预发布
- 2026-01-25: Product page 60% complete
  # 2026-01-25:产品页60%完成

## Links
## 链接
- [[Design/website-v2]]
- [[Meetings/2026-01-18-kickoff]]
\end{lstlisting}
\end{codebox}

AI读取这个就知道关于项目的一切。状态、阻碍、进度、上下文。不需要站会。不需要状态更新会议。只需问AI:``网站重设计的当前状态是什么?''

它会带着完整的上下文回答,不只是一个状态字段。

\section{基于Obsidian的业务}

经过几个月的实验,我整个业务现在都在一个Obsidian库中运行。让我向你展示结构:

\begin{codebox}
\begin{lstlisting}[style=python]
/company (Obsidian Vault)           # /公司(Obsidian库)
|
|-- /customers                      # /客户
|   |-- acme-corp.md
|   |-- techflow.md
|   `-- ... (one file per customer) # (每个客户一个文件)
|
|-- /sales                          # /销售
|   |-- /pipeline (active deals)    # /管道(活跃交易)
|   |-- /playbooks                  # /策略手册
|   |-- /templates                  # /模板
|   `-- README.md
|
|-- /marketing                      # /营销
|   |-- /content (drafts, published) # /内容(草稿,已发布)
|   |-- /campaigns                  # /活动
|   `-- /analytics                  # /分析
|
|-- /operations                     # /运营
|   |-- /orders                     # /订单
|   |-- /vendors                    # /供应商
|   `-- /inventory                  # /库存
|
|-- /finance                        # /财务
|   |-- /invoices                   # /发票
|   |-- /expenses                   # /费用
|   `-- /reports                    # /报告
|
|-- /support                        # /支持
|   |-- /tickets (open, resolved)   # /工单(打开,已解决)
|   |-- /knowledge-base             # /知识库
|   `-- /feedback                   # /反馈
|
|-- /team (AI agent configurations) # /团队(AI智能体配置)
|   |-- emma-config.md
|   |-- sam-config.md
|   |-- maya-config.md
|   |-- casey-config.md
|   |-- finn-config.md
|   `-- oscar-config.md
|
`-- /system                         # /系统
    |-- command-center.md           # 命令中心.md
    |-- daily-standup.md            # 每日站会.md
    `-- metrics.md                  # 指标.md
\end{lstlisting}
\end{codebox}

一切通过wiki风格的连接链接在一起。客户文件链接到他们的订单、发票、支持工单和通话记录。项目链接到它影响的客户。智能体配置链接到它遵循的策略手册。

当Sam成交一单时,工作流看起来像这样:

\begin{codebox}
\begin{lstlisting}[style=python]
1. Creates [[customers/new-customer.md]]
   # 创建 [[customers/new-customer.md]]
2. Updates [[sales/pipeline/this-week.md]]
   # 更新 [[sales/pipeline/this-week.md]]
3. Notifies [[team/finn-config]] to invoice
   # 通知 [[team/finn-config]] 开票
4. Updates [[system/metrics.md]]
   # 更新 [[system/metrics.md]]
5. Links to [[orders/new-order.md]]
   # 链接到 [[orders/new-order.md]]

All files update. All context preserved.
All agents can read the new state.
# 所有文件更新。所有上下文保留。
# 所有智能体可以读取新状态。
\end{lstlisting}
\end{codebox}

\section{早晨的转变}

让我向你展示我早晨例程的两个版本。

\textbf{之前(工具巡游):}

\begin{codebox}
\begin{lstlisting}[style=python]
7:00 AM - Open Salesforce, check pipeline
          # 打开Salesforce,检查管道
7:15 AM - Log into Jira, review project status
          # 登录Jira,查看项目状态
7:30 AM - Launch Tableau, look at yesterday's metrics
          # 启动Tableau,查看昨天的指标
7:45 AM - Check email in Outlook
          # 在Outlook中查看邮件
8:00 AM - Review Slack messages (12 channels)
          # 查看Slack消息(12个频道)
8:15 AM - Open Google Drive, find that document
          # 打开Google Drive,找那个文档
8:30 AM - Still not actually working
          # 还没有真正开始工作
8:45 AM - Finally start first real task
          # 终于开始第一个真正的任务

Tools opened: 6+                    # 打开的工具:6+
Time lost: 90+ minutes              # 损失的时间:90+分钟
Context switches: Too many to count # 上下文切换:多到数不清
Cognitive load: Exhausting before work begins
                                    # 认知负荷:工作开始前就累了
\end{lstlisting}
\end{codebox}

\textbf{之后(命令中心):}

\begin{codebox}
\begin{lstlisting}[style=python]
7:00 AM - Open Obsidian             # 打开Obsidian

Daily note auto-generated:          # 每日笔记自动生成:

# 2026-01-28 Tuesday
# 2026-01-28 周二

## Pipeline                         ## 管道
- $85,000 active (5 deals)          # $85,000活跃(5单)
- [[Acme Corp]] demo today at 10 AM # [[Acme Corp]]今天10点演示
- [[TechFlow]] pending proposal response
                                    # [[TechFlow]]等待提案回复

## Projects                         ## 项目
- [[Website Redesign]] - 70% complete, on track
                                    # 网站重设计 - 70%完成,正常进行
- [[Mobile App]] - Stitch generating, review tomorrow
                                    # 移动应用 - Stitch生成中,明天审核

## Yesterday's Metrics              ## 昨日指标
- Revenue: $850                     # 收入:$850
- New leads: 3 (all qualified by Sam)
                                    # 新线索:3(全部由Sam筛选)
- Support tickets: 2 (both resolved by Casey)
                                    # 支持工单:2(都由Casey解决)

## Today's Focus                    ## 今日重点
1. Acme demo at 10 AM (prep notes linked)
   # Acme 10点演示(准备笔记已链接)
2. Review website staging (link to preview)
   # 审核网站预发布(预览链接)
3. Approve Maya's blog post (draft linked)
   # 批准Maya的博客帖子(草稿已链接)

## From Your Agents (overnight)     ## 来自你的智能体(夜间)
- Sam: 3 leads qualified, 1 demo scheduled
  # Sam:3条线索已筛选,1个演示已安排
- Casey: All tickets resolved, NPS this week: 72
  # Casey:所有工单已解决,本周NPS:72
- Finn: Invoice #1048 paid ($4,500)
  # Finn:发票#1048已支付($4,500)
- Maya: Blog post ready, 2 social posts drafted
  # Maya:博客帖子准备好,2条社交帖子已起草
- Oscar: Inventory alert - reorder SKU-456
  # Oscar:库存警报 - 补货SKU-456

7:15 AM - Actually working          # 真正开始工作

Tools opened: 1                     # 打开的工具:1
Time lost: 15 minutes               # 损失的时间:15分钟
Context: Complete                   # 上下文:完整
Energy: Preserved for real work     # 精力:为真正的工作保留
\end{lstlisting}
\end{codebox}

我需要知道的一切都在一个地方。所有上下文。所有链接。我的AI团队的所有更新。一个应用。一个视图。完全清晰。

\section{成本革命}

让我分享我迁移后的真实数字。

\textbf{之前(年度企业技术栈):}

\begin{codebox}
\begin{lstlisting}[style=python]
Database (AWS RDS):           $3,600   # 数据库
Salesforce (3 seats):         $5,400   # Salesforce(3席位)
Jira + Confluence:            $3,600
Tableau (2 seats):            $1,680   # Tableau(2席位)
Microsoft 365:                $1,800
AWS Infrastructure:          $12,000   # AWS基础设施
Zapier:                         $588
Misc. SaaS:                  $11,000+  # 其他SaaS
                            --------
TOTAL:                      $39,668/year # 总计:$39,668/年

Plus: My time maintaining it (8 hrs/month)
      = 96 hours/year of admin work
# 加上:我维护的时间(8小时/月)= 96小时/年的管理工作
\end{lstlisting}
\end{codebox}

\textbf{之后(Markdown原生技术栈):}

\begin{codebox}
\begin{lstlisting}[style=python]
Obsidian Sync:                  $96    # Obsidian同步
GitHub (private repos):          $0    # GitHub(私有仓库)
Claude API:                  $1,200    # Claude API
Vercel hosting:                $240    # Vercel托管
Domain:                         $12    # 域名
Backblaze backup:               $60    # Backblaze备份
                            ------
TOTAL:                      $1,608/year # 总计:$1,608/年

Plus: My time maintaining it (1 hr/month)
      = 12 hours/year of admin work
# 加上:我维护的时间(1小时/月)= 12小时/年的管理工作
\end{lstlisting}
\end{codebox}

\textbf{年度节省:\$38,060 + 84小时}

这是真金白银和真实时间,返还给建设业务而不是维护基础设施。

\section{迁移路径}

如果你准备好进行这个转变,以下是如何在不干扰业务的情况下做到的。

\textbf{第1周:导出和理解}

从你当前的系统导出一切。先不要尝试转换——只是把数据拿出来。Salesforce到CSV。Jira到JSON。文档下载。理解你实际拥有什么,而不是你以为你拥有什么。

大多数人发现他们只使用了企业工具能力的大约百分之十。这是解放。

\begin{codebox}
\begin{lstlisting}[style=python]
Export checklist:                   # 导出清单:
- [ ] CRM: All accounts, contacts, opportunities
      # CRM:所有账户、联系人、商机
- [ ] PM tool: All projects, tasks, comments
      # PM工具:所有项目、任务、评论
- [ ] Documents: Everything, maintaining folders
      # 文档:所有,保持文件夹结构
- [ ] Email: Important threads (optional)
      # 邮件:重要线程(可选)
- [ ] Financial: Invoices, expenses, reports
      # 财务:发票、费用、报告

Revelation: Most of this data has never been read.
# 启示:这些数据大部分从未被读过。
\end{lstlisting}
\end{codebox}

\textbf{第2周:结构和转换}

用干净的文件夹结构设置你的Obsidian库。将你的导出转换为markdown。从简单开始——你随时可以增加复杂性。

\begin{codebox}
\begin{lstlisting}[style=python]
Conversion priority:                # 转换优先级:
1. Customers (most valuable context)
   # 客户(最有价值的上下文)
2. Active projects (immediate need)
   # 活跃项目(即时需要)
3. Recent deals (pipeline continuity)
   # 最近的交易(管道连续性)
4. Everything else (as needed)
   # 其他(按需)

Template tip: Create templates for each type
before converting. Consistency helps AI.
# 模板提示:在转换前为每种类型创建模板。
# 一致性有助于AI。
\end{lstlisting}
\end{codebox}

\textbf{第3周:训练和测试}

让你的AI智能体指向新库。测试查询。验证准确性。让智能体读取客户文件并总结——他们理解上下文吗?

\begin{codebox}
\begin{lstlisting}[style=python]
Agent testing:                      # 智能体测试:
- "Summarize our relationship with Acme Corp"
  # "总结我们与Acme Corp的关系"
- "What deals are closing this week?"
  # "本周有什么交易要成交?"
- "Who are our at-risk customers?"
  # "谁是我们的风险客户?"
- "Generate a weekly sales report"
  # "生成每周销售报告"

If answers are wrong, fix the data, not the AI.
# 如果答案错了,修复数据,不是AI。
\end{lstlisting}
\end{codebox}

\textbf{第4周:过渡和取消}

上线。从新系统工作几天,同时保持旧系统只读。一旦有信心,取消企业订阅。

\begin{codebox}
\begin{lstlisting}[style=python]
Cancellation sequence:              # 取消顺序:
- Day 1-3: Parallel operation (both systems)
  # 第1-3天:并行运行(两个系统)
- Day 4-5: Primary on markdown (old for reference)
  # 第4-5天:主要在markdown(旧的作为参考)
- Day 6-7: Cancel subscriptions
  # 第6-7天:取消订阅
- Month 2: Delete old accounts (after backup)
  # 第2个月:删除旧账户(备份后)

The hardest part: Believing it actually works.
# 最难的部分:相信它真的有效。
\end{lstlisting}
\end{codebox}

\section{我留下的东西}

离开企业软件感觉有风险,直到我意识到我实际上留下的是什么:

\textbf{Oracle。}我从不需要事务性数据库。Markdown表格处理我实际用数据做的一切。

\textbf{Salesforce。}我从不需要每个联系人四十七个字段。我需要的是AI能理解的上下文。

\textbf{Jira。}我从不需要故事点和速度追踪。我需要复选框和笔记。

\textbf{Tableau。}我从不需要复杂的可视化。我需要问题的答案。

\textbf{SharePoint。}我从不需要企业文档管理。我需要文件夹中的文件配合版本控制。

\section{我获得的东西}

收获让我惊讶:

\textbf{速度。}一切都是即时的。不用等待查询。不用加载仪表板。不用同步延迟。

\textbf{清晰。}我能读懂自己的数据。不用解释界面。不用抽象层。只有文本。

\textbf{可移植性。}我的业务在一个文件夹中运行。我可以把它移到任何地方。没有供应商拥有我的数据。

\textbf{AI原生。}每条信息都可以立即被AI智能体访问。不用集成项目。不用API连接器。只有文件。

\textbf{专注。}我不再是系统管理员了。我开始成为企业主。

\begin{keyinsight}[简单性悖论]
最简单的技术(纯文本)结合最先进的技术(AI)击败了中间的所有复杂性。

\textbf{Markdown + Git + AI智能体 = 现代业务技术栈}

大多数企业不需要数据库。不需要BI工具——AI生成报表。不需要PM工具——复选框和AI追踪。不需要复杂集成——AI读文件。

企业技术栈是为企业构建的。你不是企业。你是一个用AI作为团队建设有意义事物的人。

那张47,000美元的发票是发生在我业务上最好的事情。它迫使我质疑一切并发现了更好的东西。
\end{keyinsight}

\textbf{下一章:}构建用markdown文件和智能代理替代Salesforce的AI原生CRM。
