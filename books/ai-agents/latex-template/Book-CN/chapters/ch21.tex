\chapter{从数据库到第二大脑}

\section{知识管理的困境}

每位创业者都面临同样的挑战:信息散落在数十种工具中。

\begin{itemize}
\item 会议记录在 Google Docs 里
\item 任务在 Asana 或 Notion 里
\item 创意在随手记的 Apple Notes 里
\item 参考资料在 Dropbox 里
\item 联系人在电子表格里
\item 决策记录……无处可寻
\end{itemize}

结果呢?你找不到上个月写的东西。你的AI助手无法访问你的机构知识库。每一天,宝贵的洞见都在悄悄流失。

解决方案不是再找一个数据库,而是建立一个\textbf{第二大脑}——一个与你共同成长、能与AI无缝协作的统一知识系统。

\section{为什么选择 Obsidian + AI}

\subsection{Markdown 的优势}

Obsidian 是一款笔记应用,基于存储在本地电脑上的纯 Markdown 文件。这看起来很简单,但对于AI工作流来说却是革命性的:

\begin{enumerate}
\item \textbf{LLM原生格式}:Markdown 正是AI模型最擅长理解的格式
\item \textbf{本地文件}:AI代理可以直接读写,无需API
\item \textbf{面向未来}:纯文本永不过时
\item \textbf{隐私保护}:你的知识留在自己的电脑上
\item \textbf{速度快}:没有云端延迟,即时访问
\item \textbf{MCP就绪}:通过模型上下文协议服务器原生支持
\end{enumerate}

\subsection{Obsidian vs. Notion:AI视角的比较}

\begin{table}[h]
\centering
\begin{tabular}{|l|l|l|}
\hline
\textbf{特性} & \textbf{Notion} & \textbf{Obsidian} \\
\hline
数据存储 & Notion 云端 & 本地文件 \\
AI访问 & 需要API & 直接访问 + MCP \\
格式 & 专有格式 & 纯 Markdown \\
离线工作 & 有限 & 完整支持 \\
导出 & 有损导出 & 完美导出(就是文件本身) \\
AI编辑 & API调用(付费) & 免费本地访问 \\
速度 & 依赖网络 & 即时响应 \\
Claude集成 & 有限 & 通过MCP原生支持 \\
\hline
\end{tabular}
\caption{Notion vs. Obsidian AI工作流对比}
\end{table}

\subsection{文件系统即数据库}

这是一个范式转变:\textbf{你的文件系统本身就是一个数据库。}

\begin{itemize}
\item 文件夹 = 类别
\item 文件 = 记录
\item 链接 = 关系
\item 标签 = 索引
\item 文件名 = 主键
\end{itemize}

像 Claude Code 和 Claude Cowork 这样的AI代理可以通过MCP服务器或文件系统访问直接查询这个「数据库」——无需复杂的API集成。

\section{MCP革命:连接 Claude 与 Obsidian}
\index{MCP!Obsidian集成}
\index{Obsidian!MCP服务器}

\subsection{什么是MCP?}

\textbf{模型上下文协议(MCP)}\index{模型上下文协议}是一个开放标准,允许AI助手连接到外部工具和数据源。把它想象成AI的USB——一个通用连接标准。

\begin{keyinsight}[MCP改变一切]
在MCP出现之前,将AI连接到你的工具需要为每个工具进行自定义API集成。有了MCP,你只需配置一次,任何兼容MCP的AI(Claude、ChatGPT、VS Code Copilot)都可以通过标准化服务器访问你的工具。
\end{keyinsight}

\subsection{Obsidian MCP服务器}

有多个MCP服务器可以让 Claude 与你的 Obsidian 知识库交互:

\begin{enumerate}
\item \textbf{mcp-obsidian}\index{mcp-obsidian}:使用 Obsidian 的 Local REST API 插件
\item \textbf{obsidian-claude-code-mcp}:基于 WebSocket,可被 Claude Code 自动发现
\item \textbf{smithery-ai/mcp-obsidian}:读取和搜索知识库内容
\end{enumerate}

\subsection{设置 Obsidian MCP}

\begin{codebox}
\begin{lstlisting}[style=bash]
# 步骤1:安装 Obsidian REST API 插件
# 在 Obsidian 中:设置 -> 第三方插件 -> 浏览
# 搜索 "Local REST API" 并启用
# 从插件设置中复制 API 密钥

# 步骤2:安装 MCP 服务器
npm install -g mcp-obsidian

# 步骤3:配置 Claude Desktop (claude_desktop_config.json)
{
  "mcpServers": {
    "obsidian": {
      "command": "mcp-obsidian",
      "env": {
        "OBSIDIAN_API_KEY": "your-api-key",
        "OBSIDIAN_VAULT_PATH": "/path/to/vault"
      }
    }
  }
}

# 步骤4:重启 Claude Desktop
# 你的知识库现在可以被 Claude 访问了!
\end{lstlisting}
\end{codebox}

\subsection{知识操作的MCP工具}

连接后,Claude 可以使用以下 MCP 工具操作你的知识库:

\begin{codebox}
\begin{lstlisting}[style=bash]
可用的 MCP 工具
---------------

list_files_in_vault    # 浏览所有文件
list_files_in_dir      # 浏览特定文件夹
get_file_contents      # 读取笔记
search                 # 全文搜索
patch_content          # 更新现有笔记
append_content         # 追加内容到笔记
delete_file            # 删除笔记
create_note            # 创建新笔记
\end{lstlisting}
\end{codebox}

\subsection{MCP Apps:带可视界面的AI}
\index{MCP!Apps}

自2026年1月起,MCP 支持 \textbf{MCP Apps}——能够在聊天窗口内直接渲染用户界面。

\begin{codebox}
\begin{lstlisting}[style=bash]
MCP APPS 功能
-------------

当 MCP 服务器声明 UI 资源时,Claude 可以:
- 渲染交互式图表
- 显示数据输入表单
- 展示仪表板和可视化
- 呈现文件浏览器和编辑器

对于 Obsidian,这意味着:
- 笔记连接的可视化图谱
- 交互式搜索结果
- 笔记预览面板
- 标签云可视化
\end{lstlisting}
\end{codebox}

\subsection{MCP工具搜索:高效上下文管理}
\index{MCP!工具搜索}

Claude Code 的新 \textbf{MCP工具搜索}功能(2026年1月)在处理大量工具时显著减少上下文膨胀。

\begin{codebox}
\begin{lstlisting}[style=bash]
MCP 工具搜索 - 工具的延迟加载
-----------------------------

问题:加载所有 MCP 工具定义可能
消耗上下文窗口的 10% 以上。

解决方案:工具搜索创建轻量级索引。
当工具超过上下文的 10% 时:
1. 不预先加载完整定义
2. 改为构建搜索索引
3. Claude 按需搜索相关工具
4. 只加载匹配的工具

结果:token 使用减少 46.9%
     (51K tokens -> 8.5K tokens)

要求:Sonnet 4+ 或 Opus 4+ 模型
\end{lstlisting}
\end{codebox}

\section{构建你的第二大脑架构}

\subsection{文件夹结构}

\begin{codebox}
\begin{lstlisting}[style=bash]
ObsidianVault/
|-- 00-Inbox/           # 先在这里捕获一切
|-- 10-Projects/        # 活跃项目
|   |-- ProjectA/
|   |-- ProjectB/
|-- 20-Areas/           # 持续责任领域
|   |-- Business/
|   |-- Health/
|   |-- Learning/
|-- 30-Resources/       # 参考资料
|   |-- Templates/
|   |-- Snippets/
|   |-- Research/
|-- 40-Archive/         # 已完成/归档
|-- Daily/              # 每日笔记
|-- People/             # 联系人笔记(迷你CRM)
|-- Meetings/           # 会议笔记
|-- .claude/            # Claude Code 技能
|   |-- skills/         # 自定义知识库技能
\end{lstlisting}
\end{codebox}

\subsection{每日笔记作为中央枢纽}

创建一个每日笔记模板,使其成为你的指挥中心:

\begin{codebox}
\begin{lstlisting}[style=bash]
# {{date}}

## 早晨计划
- [ ] 首要任务:
- [ ] 必须完成:
- [ ] 可选完成:

## 日志
<!-- AI代理在此追加更新 -->

## 会议
<!-- 链接的会议笔记 -->

## 随手记
<!-- 一天中的快速想法 -->

## 一日总结
### 已完成
### 移至明天
### 洞见收获
\end{lstlisting}
\end{codebox}

\section{知识操作的 Claude Code 技能}
\index{Skills!知识操作}
\index{Claude Code!Skills}

\subsection{全新的 Skills 系统}

自 Claude Code 2.1.3 版(2026年1月)起,\textbf{斜杠命令已合并到 Skills 中}。Skills 更加强大——它们既可以手动调用(像命令一样),也可以在相关时由 Claude 自动调用。

\begin{codebox}
\begin{lstlisting}[style=bash]
SKILLS vs 旧斜杠命令
--------------------

旧斜杠命令:
- 仅能手动调用
- 仅限文本提示
- 没有脚本或模板

新 Skills 系统:
- 手动或自动调用
- 可包含可执行脚本
- 可捆绑模板和示例
- 跨 Claude Code、Desktop 和 Web 工作
- 优先级系统(企业 > 个人 > 项目)
\end{lstlisting}
\end{codebox}

\subsection{创建知识搜索技能}

将技能存储在知识库的 \texttt{.claude/skills/} 目录中:

\begin{codebox}
\begin{lstlisting}[style=bash]
# .claude/skills/search-brain/SKILL.md

---
name: search-brain
description: 在你的 Obsidian 第二大脑中搜索
  相关笔记、人物、会议或历史决策
---

# 搜索大脑技能

当用户询问过去的笔记、决策、人物或项目时,
使用此技能搜索知识库。

## 搜索策略(渐进式披露)

1. **先搜索文件名**(最快)
   - 在文件名中搜索查询词
   - 检查 People/、Projects/、Meetings/ 文件夹

2. **再搜索标签**
   - 在 frontmatter 中查找匹配的标签
   - 如果可用,使用 MCP 搜索工具

3. **最后搜索内容**(最慢)
   - 只有在上述方法失败时才全文搜索
   - 限制为5个最相关的结果

## 输出格式

返回找到内容的简要摘要:
- 文件路径(作为可点击链接)
- 关键摘录
- 相关笔记建议
\end{lstlisting}
\end{codebox}

\subsection{渐进式披露脚本}

在技能中包含 Python 脚本用于高级搜索:

\begin{codebox}
\begin{lstlisting}[style=python]
# .claude/skills/search-brain/search.py

import os
import subprocess
from pathlib import Path

def search_vault(query: str, vault_path: str) -> list:
    """
    渐进式披露搜索:
    快速文件名搜索 -> 较慢内容搜索
    """
    results = []
    vault = Path(vault_path)

    # 层1:文件名匹配(最快)
    for f in vault.rglob("*.md"):
        if query.lower() in f.name.lower():
            results.append(str(f))
            if len(results) >= 5:
                return results

    # 层2:内容搜索(如需要)
    if len(results) < 3:
        grep_result = subprocess.run(
            ['grep', '-ril', query, vault_path],
            capture_output=True, text=True
        )
        for match in grep_result.stdout.strip().split('\n'):
            if match and match not in results:
                results.append(match)
                if len(results) >= 5:
                    break

    return results

if __name__ == "__main__":
    import sys
    query = sys.argv[1] if len(sys.argv) > 1 else ""
    vault = os.environ.get("OBSIDIAN_VAULT", ".")
    for result in search_vault(query, vault):
        print(result)
\end{lstlisting}
\end{codebox}

\section{Claude Cowork:你的知识工作伙伴}
\index{Claude Cowork}
\index{Cowork}

\subsection{什么是 Claude Cowork?}

2026年1月12日发布,\textbf{Claude Cowork} 是 Anthropic 面向非技术知识工作的通用代理——被描述为「面向其他工作的 Claude Code」。

\begin{keyinsight}[Cowork 用于第二大脑]
Claude Code 擅长编程,而 Claude Cowork 专为创业者日常大部分知识工作设计:组织文件、处理信息、创建文档,以及维护像第二大脑这样的系统。
\end{keyinsight}

\subsection{Cowork 功能}

\begin{codebox}
\begin{lstlisting}[style=bash]
CLAUDE COWORK 功能
------------------

文件访问:
- 在指定文件夹中读取、编辑和创建文件
- 组织和重构目录
- 批量处理文档

工作方式:
- 制定计划并自主执行
- 向你通报进度
- 排队多个任务并行执行
- 感觉像给同事发消息

架构:
- 在隔离的虚拟机中运行(安全)
- 并行任务的子代理协调
- 需要时生成多个 Claude 实例
\end{lstlisting}
\end{codebox}

\subsection{Cowork 用于知识库维护}

授予 Cowork 访问你的 Obsidian 知识库文件夹:

\begin{codebox}
\begin{lstlisting}[style=bash]
COWORK + OBSIDIAN 工作流
------------------------

[你,开始新的一天]
"处理我的收件箱。分类笔记,添加标签,
 创建到相关内容的链接,归档超过30天的内容。"

[Cowork 响应]
"我现在处理你的收件箱。这是我的计划:
 1. 审查 00-Inbox 中的23条笔记
 2. 分析内容并建议类别
 3. 移动到适当的文件夹
 4. 添加标签并创建链接
 5. 归档过期材料

 完成后我会通知你。"

[30分钟后]
"收件箱处理完成:
 - 15条笔记移至 Projects/
 - 4条笔记移至 People/
 - 2条笔记已归档
 - 2条标记待你审查(内容模糊)
 - 创建了12个新的反向链接"
\end{lstlisting}
\end{codebox}

\subsection{知识工作的 Cowork 插件}
\index{Cowork!插件}

自2026年1月30日起,Cowork 支持\textbf{插件}来定制工作流。有11个预构建插件可用:

\begin{codebox}
\begin{lstlisting}[style=bash]
COWORK 插件(2026年1月)
------------------------

行业特定:
- 销售:CRM集成、提案草稿
- 财务:报告生成、数据分析
- 营销:内容日历、活动跟踪
- 法务:文档审查、合同分析

用于第二大脑:
创建自定义插件告诉 Cowork:
- 你的文件夹结构约定
- 如何处理不同笔记类型
- 首选标签分类法
- 链接和 MOC 策略
- 什么需要人工审查
\end{lstlisting}
\end{codebox}

\section{从数据库工具迁移}

\subsection{Notion 到 Obsidian 迁移}

\begin{enumerate}
\item \textbf{从 Notion 导出}:设置 $\rightarrow$ 导出 $\rightarrow$ Markdown \& CSV
\item \textbf{清理导出}:Notion 的导出包含额外格式
\item \textbf{AI辅助迁移}:使用 Claude Code 或 Cowork:
\end{enumerate}

\begin{codebox}
\begin{lstlisting}[style=bash]
# Claude Code 迁移提示:

"帮我从 Notion 迁移到 Obsidian:

1. 读取 /NotionExport 中的所有 .md 文件
2. 清理 Notion 特定格式:
   - 从文件名中删除 UUID 后缀
   - 将 Notion 属性转换为 YAML frontmatter
   - 修复损坏的内部链接
3. 组织到我的文件夹结构:
   - Projects/ 用于活跃工作
   - Archive/ 用于已完成项目
   - People/ 用于联系人页面
4. 生成迁移报告

我的 Obsidian 知识库在 ~/Documents/SecondBrain"
\end{lstlisting}
\end{codebox}

\subsection{电子表格 CRM 到人物笔记}

将你的联系人电子表格转换为链接的知识库:

\begin{codebox}
\begin{lstlisting}[style=bash]
# 联系人笔记模板:People/张三.md

---
name: 张三
company: Acme Corp
role: CTO
met: 2024-03-15
last_contact: 2025-01-20
tags: [prospect, enterprise, technical]
---

# 张三

## 背景
Acme Corp 的 CTO。在 TechCrunch Disrupt 认识。
技术背景,斯坦福 ML 博士。

## 互动记录
- [[2024-03-15]] - 会议初次见面
- [[2024-06-01]] - 演示电话,对 API 感兴趣
- [[2025-01-20]] - 跟进,Q2 预算已批准

## 笔记
- 偏好技术深入讨论而非销售演讲
- 孩子对机器人感兴趣(提到两次)
- 决策时间线:季度预算周期

## 相关链接
- [[Projects/Acme-Corp-Deal]]
- [[Meetings/2024-03-15-TechCrunch]]
\end{lstlisting}
\end{codebox}

\section{知识飞轮}

\subsection{捕获 $\rightarrow$ 处理 $\rightarrow$ 连接 $\rightarrow$ 创造}

\begin{codebox}
\begin{lstlisting}[style=bash]
第二大脑飞轮(2026版)
----------------------

1. 捕获(无摩擦输入)
   - 通过手机快速捕获 -> Claude mobile
   - 会议笔记自动转录
   - 通过浏览器扩展保存网页剪辑
   - 通过 Whisper API 转换语音备忘录
   - MCP 集成从其他应用拉取数据

2. 处理(AI辅助组织)
   - 每日:Cowork 处理收件箱
   - 每周:AI 识别孤立笔记
   - 每月:AI 建议归档候选
   - Skills 自动化常见操作

3. 连接(关系构建)
   - MCP 使 Claude 能够建议链接
   - 自动反向链接检测
   - 图谱视图揭示知识集群
   - AI 生成「内容地图」(MOC)
   - MCP Apps 可视化连接

4. 创造(输出杠杆)
   - Claude 从相关笔记起草内容
   - 通过 MCP 的检索增强生成
   - 你的声音,你的知识,AI的速度
   - 博客文章、提案、策略文档
\end{lstlisting}
\end{codebox}

\subsection{复利效应}

持续使用90天后:

\begin{itemize}
\item 第1天:空知识库,MCP 连接已配置
\item 第30天:200条笔记,Skills 自动处理收件箱
\item 第60天:500条笔记,Cowork 维护组织结构
\item 第90天:800条笔记,AI 检索变得强大
\item 第180天:你的 AI 拥有任何数据库都无法提供的机构知识
\end{itemize}

\section{进阶模式}

\subsection{品牌声音文件}

存储你的写作风格供 AI 参考:

\begin{codebox}
\begin{lstlisting}[style=bash]
# Resources/Brand-Voice.md

## 语调
- 直接但温暖
- 技术准确性重要
- 避免企业术语
- 使用具体例子

## 词汇偏好
偏好:"使用" 而非 "利用"
偏好:"帮助" 而非 "促进"
偏好:"简单" 而非 "精简"

## 结构
- 以洞见开头
- 最多支持3个要点
- 以行动项结束

## 我的写作示例
[包含3-5个你最佳写作样本]
\end{lstlisting}
\end{codebox}

通过 MCP 生成内容时:

\begin{codebox}
\begin{lstlisting}[style=bash]
# Claude 现在可以直接访问你的知识库:

"写一篇关于 AI 代理的博客文章。
 使用 get_file_contents 读取我在
 Resources/Brand-Voice.md 的品牌声音,
 然后搜索 Projects/AI-Agents-Book 中的
 相关笔记"
\end{lstlisting}
\end{codebox}

\subsection{会议智能}

将会议转化为可搜索、链接的知识:

\begin{codebox}
\begin{lstlisting}[style=bash]
# Meetings/2026-02-01-Strategy-Review.md

---
date: 2026-02-01
attendees: [[张三]], [[李四]]
project: [[Projects/Q1-Strategy]]
type: review
---

# Q1 策略审查

## 关键决策
1. 企业版提价 20%
2. 暂停功能 X 开发
3. 加倍投入集成合作伙伴

## 行动项
- [ ] [[张三]]:起草定价沟通
- [ ] [[李四]]:周五前提供合作伙伴候选名单
- [ ] 我:在 Asana 中更新路线图

## 讨论笔记
...

## 后续跟进
下次审查:[[2026-02-15-Strategy-Check-in]]
\end{lstlisting}
\end{codebox}

\section{实施路线图}

\subsection{第1周:基础搭建}

\begin{enumerate}
\item 安装 Obsidian
\item 创建文件夹结构(包括 \texttt{.claude/skills/})
\item 设置每日笔记模板
\item 安装 Obsidian REST API 插件
\item 为 Claude Desktop/Code 配置 MCP 服务器
\end{enumerate}

\subsection{第2-4周:迁移与 MCP 设置}

\begin{enumerate}
\item 从现有工具导出(Notion、Docs 等)
\item 使用 Claude Code 进行 AI 辅助迁移
\item 验证 MCP 连接正常
\item 在笔记间建立初始链接
\item 创建第一个自定义 Skills
\end{enumerate}

\subsection{第5-8周:使用 Cowork 自动化}

\begin{enumerate}
\item 设置 Claude Cowork 知识库访问
\item 配置收件箱处理自动化
\item 为常见操作创建 Skills
\item 为你的工作流构建 Cowork 插件
\item 训练检索模式
\end{enumerate}

\subsection{第9-12周:发挥杠杆}

\begin{enumerate}
\item 使用 AI 从笔记创建内容
\item 使用 MCP 工具构建自定义报告
\item 完善品牌声音集成
\item 探索 MCP Apps 可视化
\item 衡量节省的时间、获得的洞见
\end{enumerate}

\begin{keyinsight}[第二大脑的优势]
从分散的数据库转向统一的第二大脑,不是关于组织——而是关于\textbf{杠杆}。通过 MCP 将 Claude 直接连接到你的知识,Cowork 维护你的知识库,Skills 自动化常见操作,你的 AI 助手从通用工具转变为你思维的个性化延伸。

Obsidian + MCP + Claude 创造了一个随时间复利增长的知识系统。你记录的每条笔记都让 AI 更有用。你建立的每个连接都让检索更强大。你创建的每个 Skill 都让操作更快。

目标不是完美的归档系统。而是一个记住你学过的一切——并能据此采取行动的思考伙伴。
\end{keyinsight}
