\chapter{2026年AI格局}

\section{站在门槛上}

我还记得第一次请AI帮我写邮件的时候。那是2022年12月,ChatGPT刚刚发布。回复很平庸——泛泛的、有点机械、偶尔出错。但我胸口紧了一下。\textbf{我在那一刻就知道,一切都即将改变。}

三年后,我有了AI同事,他们:

\begin{itemize}
\item 处理我的电子邮件
\item 筛选我的潜在客户
\item 创作我的内容
\item 支持我的客户
\item 管理我的开票
\item 运营我的业务
\end{itemize}

2022年感觉像科幻小说的东西,到2026年已经是稀松平常。而我们才刚刚开始。

本章是我的领域地图——不是每个AI工具的详尽百科全书,而是一个实用指南,告诉你什么对构建一人公司真正重要。我们将探索主要玩家,了解他们的优势,最重要的是,弄清楚每个工具在你业务中的位置。

\section{我们见证的进化}

变化的速度令人眩晕。让我把它放到透视中:

\begin{itemize}
\item \textbf{2023年}:我们获得了对话式AI。你可以与模型聊天,它会连贯地回应。革命性的,但有限——它能说话,但不能做事。

\item \textbf{2024年}:我们获得了能使用工具的AI。模型学会了调用API、搜索网络、运行代码。它们从对话者变成了助手。但仍然需要你来设置一切。

\item \textbf{2025年}:我们获得了能操作计算机的AI。Claude的Computer Use功能是突破。AI第一次可以看到你的屏幕、移动鼠标、点击按钮。它可以使用你能使用的任何软件——不需要API。

\item \textbf{2026年}:我们有了AI同事。不是你使用的工具,而是与你并肩工作的队友。他们有自己的上下文、自己正在进行的任务、自己的责任领域。
\end{itemize}

从"AI作为工具"到"AI作为同事"的转变听起来可能只是语义上的区别,但它改变了你构建业务的一切。

\section{Claude:深思熟虑的同事}

我在这里有偏见,我会坦诚:Claude是我的主要AI伙伴。我广泛使用过所有主要模型,Claude是我一直回来使用的那个。让我解释原因,然后你可以决定这些原因是否适用于你的情况。

\subsection{Claude的进化}

Anthropic在2023年发布了Claude 2,专注于安全性和有用性。它很好,但GPT-4在原始能力上更强。然后在2024年初Claude 3问世——Opus、Sonnet和Haiku——游戏改变了。尤其是Sonnet,成为我用过的最好的编程模型。它对上下文的理解比市场上任何其他模型都好。

2024年晚些时候,Claude 3.5 Sonnet到来,我围绕它重写了我一半的业务流程。推理能力和编程熟练度的结合是无与伦比的。然后Computer Use来了——Claude能够看到并与屏幕交互的能力。

到2025年,Anthropic发布了Claude Code,一个驻留在你终端中的命令行工具,可以阅读、编写和重构整个代码库。他们随后推出了Agent SDK,使构建自定义AI智能体变得简单。在2026年,Claude 4带来了扩展的上下文、实时协作和跨会话记忆。

\subsection{Claude的不同之处}

我发现Claude在三个领域表现出色:

\textbf{谨慎推理。}Claude在行动前思考。当我给它一个复杂问题时,它会考虑多个角度,承认不确定性,并解释其思考过程。这对于商业运营非常重要,因为自信但错误的答案比没有答案更糟糕。

\textbf{长上下文理解。}凭借20万+词元的上下文窗口,Claude可以在一次对话中阅读和理解整个知识库。我可以向它展示我完整的操作手册库、所有客户上下文、我完整的代码库——它会综合所有这些信息。

\textbf{Computer Use。}这是游戏规则改变者。Claude可以看到并与任何软件界面交互。当我需要在没有API的遗留软件中自动化工作流程时,Claude可以像人类一样使用界面。它浏览网站、填写表单、点击按钮、阅读仪表板。

\subsection{Claude Code:开发者的梦想}

如果你做任何编程——即使只是配置你的智能体——Claude Code值得了解。

用Homebrew安装它:

\begin{codebox}
\begin{lstlisting}[style=bash]
brew install claude-code
\end{lstlisting}
\end{codebox}

然后直接从终端使用它:

\begin{codebox}
\begin{lstlisting}[style=bash]
claude "add authentication to this app using Firebase"
\end{lstlisting}
\end{codebox}

接下来发生的事情仍然让我惊叹。Claude阅读你的代码库,理解其结构,在多个文件中进行协调的更改,运行你的测试,并修复任何问题。我曾让它重构整个应用程序、生成全面的文档,以及在凌晨2点调试生产问题。

对于一人公司来说,这就像有一个全天候待命的高级开发人员。

\subsection{Claude最适合什么}

根据我的经验,Claude非常适合:

\begin{itemize}
\item 行政助理任务(电子邮件、日历、会议准备)
\item 客户成功运营(理解上下文、做出细微决策)
\item 需要推理的复杂操作
\item 软件开发和技术工作
\item 任何需要谨慎、深思熟虑回应的事情
\end{itemize}

\section{Gemini:多模态强者}

Google的Gemini采取了不同的方法。Claude擅长对文本进行深度推理,而Gemini从一开始就被设计为理解多种模态:文本、图像、视频和音频作为原生输入。

\subsection{原生多模态}

这在实践中意味着什么呢?我可以向Gemini发送一个制造过程的视频,让它识别质量问题。我可以给它一段销售电话录音,获得带有行动项的结构化摘要。我可以分享一个仪表板的截图,让它分析异常。

这不是一个处理文本并碰巧接受图像的模型。它在视觉上、空间上、时间上思考。当你向它展示视频时,它理解运动、序列和随时间的变化。

\subsection{Google Workspace集成}

如果你的业务运行在Google上——Gmail、Docs、Sheets、Calendar、Meet——Gemini的集成是无缝的。它不是你调用的单独工具;它嵌入在你已经使用的应用程序中。

在Gmail中,它起草回复并总结长线程。在Docs中,它帮助研究和生成内容。在Sheets中,它分析数据、创建公式并发现洞见。在Meet中,它提供实时笔记,越来越多地还有翻译。

这种零摩擦的方法对于采用很重要。你不需要改变你的工作流程;AI就出现在你已经工作的地方。

\subsection{NotebookLM:被低估的宝石}

让我告诉你关于NotebookLM的事情,因为我认为它是Google最被低估的产品之一。

上传文档到NotebookLM,Gemini就成为那些材料的专家。上传你的员工手册,你就有了一个入职助理,可以回答关于你政策的任何问题。上传你的产品文档,你就有了一个面向客户的知识库。上传研究论文,你就有了一个综合引擎,可以在来源之间找到联系。

我用它来准备会议。在任何重要通话之前,我上传所有相关文档——以前的通信、产品规格、竞争分析——然后花十分钟提问。我带着比以往任何时候都更充分的准备走进会议。

\subsection{Gemini最适合什么}

Gemini在以下方面表现出色:

\begin{itemize}
\item 文档处理和分析
\item 多模态任务(视频、图像、音频)
\item 会议协助和实时笔记
\item Google用户的工作空间自动化
\item 无缝集成比原始能力更重要的任务
\end{itemize}

\section{OpenAI:先驱者}

OpenAI开启了这场革命。ChatGPT让AI对世界来说变得可访问。GPT-4展示了这些模型真正能做什么。虽然我已经把大部分工作转移到了Claude,但OpenAI的生态系统仍然是最大和最成熟的。

\subsection{OpenAI的优势}

OpenAI的主要优势是生态系统的广度。与OpenAI集成的工具比任何其他提供商都多。更多的开发人员有使用他们API的经验。更多的教程存在。更多的例子比比皆是。

如果你正在寻找阻力最小的路径——与其他一切"就是能用"的AI堆栈——OpenAI通常是答案。

\subsection{Custom GPTs:零代码智能体}

Custom GPTs是OpenAI最易访问的创新。无需编写任何代码,你可以:

\begin{enumerate}
\item 定义智能体的名称和目的
\item 上传它可以参考的知识文件
\item 启用功能(网页浏览、代码解释、图像生成)
\item 设置对话启动语以引导用户
\item 通过简单的链接分享
\end{enumerate}

我见过企业构建客户FAQ机器人、产品专家、写作助手和内部流程指南——全都不需要接触代码。对于非技术创始人来说,这通常是获得工作AI智能体的最快路径。

\subsection{Operator:浏览器自动化}

OpenAI的Operator是他们对Claude Computer Use的回应。它可以自主浏览网页、填写表单、完成交易并处理多步骤工作流程。

我用它来完成研究任务——收集竞争情报、从多个来源收集数据、自动化表单提交。根据我的经验,它不如Claude的Computer Use精确,但正在快速改进。

\subsection{OpenAI最适合什么}

选择OpenAI用于:

\begin{itemize}
\item 最大的集成生态系统
\item 通过Custom GPTs实现零代码智能体
\item Microsoft环境(通过Azure OpenAI)
\item 带有DALL-E集成的内容创作
\item 语音应用
\end{itemize}

\section{开源运动}

不是每个人都想依赖封闭的API。对于一些企业来说,运行自己的AI模型不仅是偏好——而是必需的。

\subsection{Meta的Llama}

Meta的Llama 3改变了开源游戏。凭借宽松的许可和接近封闭模型的质量,它打开了以前被锁住的大门。

你可以在自己的服务器上、自己的云中、按自己的条款运行Llama。没有API调用离开你的网络。没有数据发送给第三方。完全控制。

权衡是运营复杂性。你需要基础设施、专业知识和持续维护。但对于隐私敏感的应用、受监管的行业或大规模的纯粹成本优化,这是值得的努力。

\subsection{Mistral}

Mistral,这家法国AI初创公司,专注于效率。他们的模型在每个参数上提供卓越的质量,使其非常适合资源受限的部署。

对于担心数据主权的欧洲企业,Mistral提供了一个有吸引力的替代方案——一个来自欧盟内部的有能力的模型,其开发中内置了欧洲价值观。

\subsection{DeepSeek}

DeepSeek从中国崛起,拥有开放权重和非常有竞争力的质量。他们的R1推理模型以极低的成本与最好的封闭模型竞争。

我用DeepSeek处理高容量、成本敏感的工作负载。当你每天处理数百万个词元时,价格差异会显著累积。

\subsection{何时选择开源}

决策树很简单:

\begin{itemize}
\item \textbf{严格的隐私要求?}开源,自托管。
\item \textbf{每天超过一百万个词元?}计算成本差异。
\item \textbf{需要为特定任务微调?}开源是你唯一的选择。
\item \textbf{想避免供应商锁定?}开源提供灵活性。
\item \textbf{刚刚起步?}封闭API更简单。从那里开始。
\end{itemize}

\section{中国AI:平行宇宙}

虽然西方AI发展占据了大多数头条,但中国一直在建设强大的能力。

阿里巴巴的通义千问模型提供出色的中文语言支持和良好的英语能力。01.AI的Yi模型提供长上下文和强大的推理能力。Moonshot的Kimi专门从事文档处理,具有显著的上下文窗口。DeepSeek,已经提到过,提供一些最具成本效益的推理。

对于服务中国市场的企业,这些模型通常是正确的选择。它们理解文化细微差别,原生处理中文文本,并提供有竞争力的定价。

但要仔细考虑影响。美国政府合同、敏感数据处理和某些监管环境可能使中国AI模型不适用,无论其能力如何。

\section{垂直AI:领域专家}

有时候,通用AI不够。垂直AI公司在特定领域建立深厚的专业知识。

\subsection{Harvey用于法律}

Harvey改变了法律工作。以前需要律师助理花费数天的合同审查现在需要数小时。以前需要昂贵数据库的法律研究现在以对话方式进行。以前需要大量初级律师的尽职调查现在只需要一小部分团队。

如果你在法律行业,看看Harvey。领域专业知识是非凡的。

\subsection{Hippocratic AI用于医疗保健}

Hippocratic AI专注于安全关键的医疗保健应用。患者沟通、预约安排、就诊前准备、就诊后跟进——所有这些都由专门为医疗环境设计的AI处理。

FDA关注和安全优先的方法在医疗保健中很重要。你不希望通用AI提供医疗建议。

\subsection{GitHub Copilot用于开发}

如果你写代码,你可能已经在使用GitHub Copilot了。凭借IDE集成、代码补全、聊天协助和测试生成,它已成为开发人员的标准装备。

GitHub的数据显示编码速度提高55\%。这与我的经验相符。

\subsection{Sierra用于客户服务}

由前Salesforce和前Google高管创立,Sierra构建企业级客户服务智能体。完整的对话处理、多渠道支持、CRM集成和智能人工升级。

如果客户服务是你的瓶颈,Sierra值得评估。

\section{MCP:AI的USB}

这是我在AI旅程早期遇到的问题:每个工具都需要与每个AI进行自定义集成。如果你有五个AI工具和十个业务应用程序,你需要五十个不同的集成。这是一场噩梦。

Anthropic的模型上下文协议(MCP)改变了这一点。

MCP是连接AI与工具的开放标准。有了MCP,工具实现一次协议,AI实现一次,一切都连接起来。一个标准接口,通用兼容性。

把它想象成AI的USB。在USB之前,每个设备需要不同的电缆、不同的端口、不同的驱动程序。USB标准化了连接,突然间一切都与一切兼容。

MCP对AI集成做了同样的事情。我可以使用MCP兼容的工具与任何MCP兼容的AI配合。当我更换AI提供商时,我的集成仍然有效。当新工具出现时,它们与我现有的AI配合使用。

如果你是为长期建设的,优先考虑MCP兼容的集成。它们是面向未来的,而自定义集成不是。

\section{选择你的技术栈}

有了所有这些选择,你如何选择?让我给你我的框架。

\subsection{从你的主要工作开始}

你大部分时间在做什么?

\begin{itemize}
\item \textbf{文本和推理:}从Claude开始
\item \textbf{视觉和多模态:}从Gemini开始
\item \textbf{Microsoft生态系统:}从OpenAI开始
\item \textbf{代码生成:}从Claude Code开始
\item \textbf{隐私关键:}从Llama或Mistral开始
\item \textbf{中国市场:}从通义千问或Yi开始
\end{itemize}

\subsection{考虑你的预算}

你的月度AI预算决定了你的选择:

\begin{itemize}
\item \textbf{\$0(免费层):}Claude Free、Gemini Free、ChatGPT。足够学习和实验。
\item \textbf{\$20-50/月:}Claude Pro、ChatGPT Plus、Gemini Advanced。认真的个人生产力。
\item \textbf{\$100-500/月:}跨多个提供商的API访问。真正的自动化开始。
\item \textbf{\$500+/月:}企业功能、多个智能体、显著规模。
\end{itemize}

\subsection{我推荐的2026年技术栈}

根据我的经验,这是我的推荐:

对于开发,使用Claude Code作为你的主要工具。添加Cursor(配合Claude)用于IDE集成。移动开发通过Stitch处理。

对于业务运营,Claude通过API驱动你的智能体。Gemini和NotebookLM处理文档智能。Google的AI功能自动化你的工作空间。

对于成本敏感的工作负载,DeepSeek处理高容量处理。微调的Llama模型覆盖专业任务。Gemini Flash或Claude Haiku处理边缘情况。

对于企业,Claude和Anthropic提供你的主要基础设施。Azure OpenAI覆盖Microsoft环境。Gemini Enterprise处理Google生态系统。

\section{接下来会发生什么}

变化的步伐不会放慢。这是我预期的:

到2026年底,智能体协作将是正常的。大多数知识工作将有AI协助。Computer Use将从令人印象深刻变为可靠。开源将缩小与封闭模型的大部分质量差距。

到2027年,自主智能体将在没有监督的情况下处理日常任务。多智能体系统将在生产环境中协调。AI原生企业将主导其类别。传统SaaS将面临颠覆。

到2028年,AGI辩论将加剧。监管将明确化。AI智能体将像网站一样普遍。人类-AI协作将成为标准,而不是例外。

在那之后,我不会假装预测。变革将太深刻,变化将太根本。

\section{你的行动计划}

你应该如何处理所有这些信息?这是我的建议:

\textbf{本月:}选择你的主要AI工具。学习有效地提示。构建你的第一个工作流程自动化。跟踪你节省的时间。

\textbf{本季度:}实施你的第一个智能体。将其连接到你的业务系统。记录你的操作手册。衡量影响。

\textbf{今年:}建立你完整的智能体团队。过渡到AI原生运营。发展竞争优势。无需招聘即可扩展。

\textbf{持续:}保持对能力的了解。随着AI改进升级你的智能体。适应新的可能性。引领,而不是跟随。

\begin{keyinsight}[战略现实]
AI格局每月都在变化。你今天选择的工具可能不是你一年后使用的工具。但你发展的技能——提示工程、操作手册创建、智能体管理——可以跨任何平台转移。投资于能力,而不仅仅是工具。
\end{keyinsight}

\textbf{下一章:}现在你理解了这个格局,是时候认识你的第一个AI同事了。Emma,你的行政助理,将改变你处理电子邮件、日历和运营业务的日常混乱的方式。
