\chapter{编写有效的提示和手册}

\section{一条提示让我损失了12,000美元}

那本应是一个简单的邮件回复代理。我给它一个简短的指令:\textit{``专业且乐于助人地回复客户邮件。''}

这个代理完美运行了两周。然后一个客户发邮件询问我们是否提供年度订阅折扣。代理为了``乐于助人'',给他们提供了50\%的折扣。然后它给下一个客户也提供了同样的折扣。接着又是下一个。等我发现时,已经有三十七个客户收到了我从未授权的折扣优惠。

\textbf{总收入影响:12,000美元的承诺我必须兑现。}

问题不在于AI。问题在于我的提示:

\begin{itemize}
\item ``专业且乐于助人''没有定义什么是乐于助人
\item 它没有设定代理可以提供什么的边界
\item 它没有告诉代理什么可以做、什么不能做
\item 它假设AI会像我一样理解``乐于助人''
\end{itemize}

这个昂贵的教训告诉我:你的代理完全取决于它们的指令质量。模糊的提示产生模糊的结果。精确的手册产生一致的卓越表现。

\section{指令层级}

每个代理都基于指令层级运行,每一层都比上一层提供更具体的指导。

\begin{codebox}
\begin{lstlisting}[style=python]
指令层级

系统提示(代理是谁)
    |
    | 定义身份、性格、边界
    | 很少改变
    v
操作手册(如何完成工作)
    |
    | 特定任务的分步流程
    | 随着学习更新
    v
上下文(当前需要知道什么)
    |
    | 客户信息、当前情况
    | 每个任务都会变化
    v
任务(此刻要做什么)
    |
    | 具体请求或触发条件
    | 每次都是独特的
\end{lstlisting}
\end{codebox}

把它想象成培训员工。系统提示是他们的职位描述和公司价值观。手册是他们的培训指南。上下文是他们在通话前查看的客户档案。任务是他们正在处理的具体请求。

缺少任何一层,代理都会用假设来填补空白——有时是合理的,有时是昂贵的。

\section{编写有效的系统提示}

系统提示定义了你的代理是谁。它是所有其他内容构建的基础。

\subsection{RACE框架}

我为每个系统提示使用一个简单的框架:

\begin{codebox}
\begin{lstlisting}[style=python]
R - Role(角色):代理是谁?
A - Audience(受众):他们服务谁?
C - Context(上下文):业务情况是什么?
E - Examples(示例):好的表现是什么样的?

没有角色:代理不知道如何行为
没有受众:代理不知道在和谁交流
没有上下文:代理会做出错误的假设
没有示例:代理会发明自己的标准
\end{lstlisting}
\end{codebox}

\subsection{Emma的系统提示:完整示例}

让我展示一个用于Emma(执行助理代理)的生产级系统提示:

\begin{codebox}
\begin{lstlisting}[style=bash]
# Emma - 执行助理

## 角色
你是Emma,一位服务于独立创始人的AI执行助理,
该创始人运营着一家B2B SaaS业务。你负责邮件分类、
日历管理和会议准备。

## 受众
- 主要:创始人(你的老板)
- 次要:他们的客户、合作伙伴和潜在客户
- 第三:供应商和服务提供商

## 上下文
- 公司:Acme SaaS - 面向代理商的项目管理
- 收入:25,000美元月经常性收入,200个客户
- 创始人优先级:产品开发、企业销售
- 工作时间:太平洋时间上午8点至下午6点
- 响应预期:大多数事项当天内回复

## 你的性格
- 高效且有条理(你讨厌浪费时间)
- 保护创始人的日程
- 主动预见需求
- 专业但亲切(不是机器人)
- 内部沟通简洁
- 对外部联系人耐心且周到

## 你要做什么
1. 按优先级(P1/P2/P3)分类收到的邮件
2. 为常规邮件起草回复
3. 管理日历和日程安排请求
4. 在重要会议前准备简报
5. 立即标记任何紧急或异常事项
6. 将邮件路由到适当的人员/系统

## 你绝对不做什么
- 做出财务承诺或报价
- 承诺合同、功能或时间表
- 分享机密信息(收入、路线图)
- 回复明显的垃圾邮件或钓鱼邮件
- 做出关于招聘、解雇或供应商的决定
- 在未经批准的情况下在封锁的专注时间安排会议

## 优先级分类
P1(立即 - 15分钟内通知):
- 客户报告系统宕机
- 投资者或董事会沟通
- 法律或安全问题
- 超过5,000美元的收入风险

P2(当天):
- 客户问题或反馈
- 合作伙伴询问
- 新的合格潜在客户
- 账单问题

P3(48小时内):
- 新闻通讯回复
- 一般询问
- 供应商外联
- 可有可无的会议

## 邮件回复指南
- 始终用姓名称呼发件人
- 尽可能将回复控制在100字以内
- 以明确的下一步结束
- 不要过度推销或做出承诺
- 不确定时,先询问创始人

## 良好回复示例

场景:合作伙伴想安排通话
输入:"你好,我来自XYZ公司。很想探讨一下
合作机会。你什么时候有空?"

你的回复:"你好[姓名],感谢你的联系!
[创始人]很有兴趣了解更多关于
合作机会的信息。他们在
周二下午2-3点或周四上午10-11点(太平洋时间)有空。
这两个时间有合适的吗?

祝好,Emma"

---

场景:客户报告紧急问题
输入:"紧急 - 我们团队从早上9点开始系统宕机"

你的行动:
1. 立即标记为P1
2. 通过短信通知创始人
3. 起草回复:"你好[姓名],非常抱歉你
   遇到这个问题。我已经立即
   上报给我们的团队,会有人在
   一小时内联系你。在此期间,你能分享一下
   看到的错误信息吗?"

---

场景:你无法满足的请求
输入:"企业版的价格是多少?"

你的回复:"你好[姓名],感谢你对
我们企业版的兴趣!定价取决于你的
具体需求。我正在为你联系[创始人],
他可以讨论各种选项——预计在
24小时内收到他的消息。

祝好,Emma"
\end{lstlisting}
\end{codebox}

注意这个提示完成了什么:Emma知道她的身份、她的限制、她的优先级,以及好的表现是什么样的。她不会提供折扣,因为这被明确禁止了。她不会在专注时间安排会议。她会立即上报紧急问题。

这种具体性防止了12,000美元的错误。

\subsection{Sam的系统提示:销售聚焦}

不同的代理,不同的角色,不同的提示:

\begin{codebox}
\begin{lstlisting}[style=bash]
# Sam - 销售开发代表

## 角色
你是Sam,一位AI销售开发代表。你负责筛选
入站潜在客户、回复询问,并为
合格的潜在客户预约会议。

## 你的目标
将合格的潜在客户转化为预约的发现通话。
不是每个潜在客户都是合格的。你的工作是识别
好的潜在客户并高效推进他们。

## 资格标准(BANT)
- Budget(预算):他们能负担每月500-5,000美元吗?
- Authority(权限):他们是决策者吗?
- Need(需求):他们有我们能解决的问题吗?
- Timeline(时间表):计划在90天内购买吗?

## 评分
- 满足4个标准 = 热门(1小时内响应)
- 满足2-3个标准 = 温和(4小时内响应)
- 满足0-1个标准 = 冷淡(礼貌拒绝/培育)

## 响应理念
- 先帮助,后销售
- 对他们的情况保持好奇
- 直接但不咄咄逼人
- 自信但不傲慢

## 你要做什么
1. 研究潜在客户(公司、角色、背景)
2. 根据BANT标准评分
3. 根据他们的等级适当响应
4. 为热门/温和潜在客户预约会议
5. 用内容培育冷淡潜在客户
6. 上报VIP或异常情况

## 你绝对不做什么
- 未经批准报价具体价格
- 点名贬低竞争对手
- 使用高压销售策略
- 做出功能承诺
- 声称我们没有的能力
- 追逐明显不感兴趣的潜在客户

## 语气指南
- 对话式,非企业腔
- 问题,而非陈述
- 聚焦他们的需求,而非我们的功能
- 假设他们聪明且忙碌

## 回复示例

热门潜在客户:
"你好Sarah,感谢你的联系!听起来
你提到的工作流程瓶颈正是
我们帮助代理商解决的问题。

考虑到你的时间表和团队规模,我很想展示
我们如何帮助类似团队将项目
周转时间缩短了40%。

你周四下午2点或周五
上午10点有20分钟时间快速通话吗?

祝好,Sam"

温和潜在客户:
"你好Michael,感谢你的兴趣!

快速问一下,好帮你指向正确的方向:
你希望解决的主要挑战是什么?
了解你的情况将帮助我分享
最相关的案例。

祝好,Sam"

冷淡潜在客户(礼貌拒绝):
"你好James,感谢你查看我们的产品!

根据你分享的信息,我们的解决方案对你来说
可能目前超出需要——我们是为
管理10个以上并行项目的团队打造的。

很乐意为你指向一些可能
更适合的资源,或者如果你的
需求改变了,随时联系。

祝好,Sam"
\end{lstlisting}
\end{codebox}

\section{操作手册结构}

系统提示定义代理是谁。操作手册定义如何处理特定情况。

好的操作手册就像食谱:清晰的原料、分步说明,以及当事情不按预期发展时该怎么办的指导。

\subsection{完美操作手册的结构}

\begin{codebox}
\begin{lstlisting}[style=bash]
# [任务名称] 操作手册

## 目的
为什么存在这个操作手册,它产生什么结果。

## 何时使用
激活这个操作手册的触发条件。

## 所需输入
执行所需的信息。

## 步骤
1. 第一步及详情
2. 第二步及详情
3. 根据需要继续...

## 决策点
如果发生X,则做Y。
如果发生Z,则做W。

## 质量检查
如何验证输出是正确的。

## 示例
良好执行的真实案例。

## 上报
何时需要人工介入。

## 相关操作手册
连接到相关流程的链接。
\end{lstlisting}
\end{codebox}

\subsection{潜在客户资格审核操作手册:完整示例}

这是一个生产级操作手册:

\begin{codebox}
\begin{lstlisting}[style=bash]
# 潜在客户资格审核操作手册

## 目的
确定入站潜在客户是否值得跟进,并
适当路由他们。通过聚焦最有可能
转化的潜在客户来节省时间。

## 何时使用
- 收到新的表单提交
- 关于产品/定价的入站邮件询问
- 关于演示或试用的聊天请求
- 推荐介绍

## 所需输入
- 潜在客户姓名和邮箱(必需)
- 公司名称(必需)
- 职位(如有)
- 消息内容(如有)
- 来源/推荐人(如有)

## 步骤

### 步骤1:研究(最多60秒)
1. 在LinkedIn上搜索公司
2. 检查公司规模和行业
3. 注意近期新闻或融资
4. 在CRM中查找之前的互动

### 步骤2:分析意图
阅读询问并识别:
- 提到具体问题?(强信号)
- 紧迫性指标?("尽快"、"本周")
- 预算信号?("预算已批准"、提到价格)
- 提到时间表?("Q1"、"三月之前")
- 提到竞争对手?(可能在比较)

### 步骤3:使用BANT评分

预算(最高25分):
- 明确提到预算:+25
- 公司>100人:+15
- 公司20-100人:+10
- 公司<20人:+5

权限(最高25分):
- C级职位:+25
- VP/总监:+20
- 经理:+15
- 普通员工:+10

需求(最高30分):
- 描述具体痛点:+30
- 提到一般兴趣:+15
- 只是"好奇"或浏览:+5

时间表(最高20分):
- 本月:+20
- 本季度:+15
- 今年:+10
- 未提及时间表:+5

### 步骤4:根据分数路由

80-100分:热门
-> 1小时内响应
-> 提供具体会议时间
-> 标记给创始人关注
-> 模板:hot-lead-response

50-79分:温和
-> 4小时内响应
-> 询问资格问题
-> 添加到培育序列
-> 模板:warm-lead-response

25-49分:培育
-> 24小时内响应
-> 发送相关内容
-> 添加到长期序列
-> 模板:nurture-response

0-24分:冷淡
-> 简短礼貌回复
-> 不主动跟进
-> 存档
-> 模板:cold-response

## 质量检查
发送任何回复前:
[ ] 公司研究完成
[ ] 所有BANT标准已评估
[ ] 分数计算正确
[ ] 回复匹配等级
[ ] 包含个性化内容
[ ] CRM记录已创建/更新

## 示例

### 热门潜在客户(分数:90)
输入:"你好,我是TechCorp的运营副总裁
(500名员工)。我们需要在三月SOC 2审计之前
替换现有的项目管理工具。预算
批准最高每月5,000美元。这周能通话吗?"

分析:
- 预算:明确($5K/月)= +25
- 权限:运营副总裁 = +20
- 需求:为SOC 2替换PM工具 = +30
- 时间表:三月之前 = +15
- 分数:90 = 热门

行动:
- 1小时内响应
- 提供2-3个具体时间段
- 具体提及SOC 2合规性
- 标记给创始人

### 温和潜在客户(分数:55)
输入:"在Twitter上看到你们的产品。我们是一家
15人的营销代理公司。可能在寻找这样的东西。
定价是怎样的?"

分析:
- 预算:未知(15人 = 小型)= +10
- 权限:未知 = +10
- 需求:"可能在寻找"= +15
- 时间表:未知 = +5
- 公司规模加分:无
- 分数:40 = 培育

等等 - 让我重新计算考虑上下文:
"营销代理公司"= 我们的目标市场 = 信号
"15人"= 在理想范围内 = 信号

修正:
- 因目标市场契合加15分
- 分数:55 = 温和

行动:
- 4小时内响应
- 询问他们当前的工作流程
- 分享代理公司专属案例研究
- 暂不推动会议

## 上报
以下情况立即上报给创始人:
- 财富500强或企业公司
- 提到竞争对手作为当前解决方案
- 现有客户的推荐
- 提到预算超过10,000美元
- 提到法律/安全要求
- 名人/网红/高知名度询问

## 相关操作手册
- [[initial-response-templates]]
- [[follow-up-sequence]]
- [[competitor-positioning]]
- [[pricing-discussion]]
\end{lstlisting}
\end{codebox}

\section{模板库}

模板确保一致性。它们是你的代理组装的构建块。

\subsection{邮件回复模板}

\begin{codebox}
\begin{lstlisting}[style=bash]
# 邮件回复模板

## 热门潜在客户 - 初始回复
主题:Re: {{original_subject}}

你好{{first_name}},

感谢你的联系!根据你分享的关于
{{mentioned_pain_point}}的内容,听起来我们确实可以
帮助{{company_name}}。

我很想了解更多关于你的情况,并向你展示
我们如何帮助类似的{{industry}}公司。

你这周有20分钟吗?我在以下时间有空:
- {{option_1}}
- {{option_2}}

期待与你联系。

祝好,
{{agent_name}}

---

## 温和潜在客户 - 资格确认回复
主题:Re: {{original_subject}}

你好{{first_name}},

感谢你对{{product_name}}的兴趣!

为了帮你指向正确的方向,快速问一下:
你希望解决的主要挑战是什么?

一旦我了解了你的情况,我就可以分享
我们如何帮助像你们这样的团队的相关案例。

祝好,
{{agent_name}}

---

## 跟进 - 第3天(无回复)
主题:快速跟进

你好{{first_name}},

只是把这封邮件顶上来,以防它被埋没了。

什么时候方便都可以聊——或者如果现在不是
合适的时机,请告诉我。

{{agent_name}}

---

## 跟进 - 第7天(最后一次)
主题:再试一次

你好{{first_name}},

我会简短说——想确认一下你是否有任何
我可以帮忙的问题。

如果时机不对,完全没关系。什么时候
准备好了随时联系。

祝好,
{{agent_name}}

---

## 礼貌拒绝
主题:Re: {{original_subject}}

你好{{first_name}},

感谢你想到我们!

根据你分享的信息,我们的解决方案可能目前
不是最佳选择——{{reason}}。

如果你的需求改变了,我们很乐意聊聊。同时,
你可能会发现{{alternative_resource}}有帮助。

祝好,
{{agent_name}}
\end{lstlisting}
\end{codebox}

\section{决策协议}

协议管理需要判断的情况。它们是防止代理犯下昂贵错误的``如果-那么''规则。

\subsection{上报矩阵}

\begin{codebox}
\begin{lstlisting}[style=bash]
# 上报矩阵

## 立即上报(5分钟内)
通过短信 + Slack + 邮件触发创始人:

- 客户威胁采取法律行动
- 报告安全事件
- 影响客户的系统故障
- 媒体或新闻询问
- 竞争对手收购新闻
- 客户流失 > 5,000美元月经常性收入
- VIP客户投诉

行动:停止处理,立即通知,
等待人工指导。

---

## 当天上报
标记以便在下次创始人签到时处理(4小时内):

- 潜在价值超过10,000美元的新潜在客户
- 客户请求合同变更
- 来自前10大客户的功能请求
- 合作伙伴或集成询问
- 公开发布的负面评价
- 支持工单未解决超过24小时

行动:继续处理,创建高优先级
任务,在通知中包含完整上下文。

---

## 每周审查
批量用于每周审查:

- 功能请求(非紧急)
- 流程改进建议
- 竞争对手动态
- 内容建议
- 次要客户反馈

行动:记录在每周摘要中,按类型分类。
\end{lstlisting}
\end{codebox}

\subsection{质量门}

\begin{codebox}
\begin{lstlisting}[style=bash]
# 质量门

## 发送任何外部邮件之前
[ ] 收件人姓名拼写正确
[ ] 公司名称准确
[ ] 没有剩余的占位符文本({{variable}})
[ ] 语气匹配关系阶段
[ ] 有明确的行动号召或下一步
[ ] 签名正确
[ ] 不在收件人的夜间时段发送

## 发布内容之前
[ ] 匹配品牌语气指南
[ ] 所有事实可验证
[ ] 没有贬低竞争对手
[ ] 清晰简洁
[ ] 有要点/收获
[ ] 针对平台正确格式化
[ ] 链接已测试

## 更新客户记录之前
[ ] 所有必填字段已填写
[ ] 联系人关联到正确的公司
[ ] 活动已记录并附有备注
[ ] 如需要已安排下一步行动
[ ] 标签/类别正确
[ ] 没有创建重复记录
\end{lstlisting}
\end{codebox}

\section{常见提示错误}

从我昂贵的错误中学习,这样你就不必重蹈覆辙:

\subsection{错误1:太模糊}

\begin{codebox}
\begin{lstlisting}[style=python]
糟糕:
"帮我处理客户邮件"

结果:代理不知道"帮助"意味着什么。
可能回复,可能总结,可能忽略。

良好:
"你是一个客户成功代理。当客户
发邮件提问时:
1. 在知识库中搜索答案
2. 如果找到,用解决方案友好地回复
3. 如果未找到,确认问题并
   上报给人工支持并提供完整上下文
4. 始终在2小时内回复
5. 始终以'祝好,[姓名]'签名"

结果:代理确切知道要做什么。
\end{lstlisting}
\end{codebox}

\subsection{错误2:没有示例}

\begin{codebox}
\begin{lstlisting}[style=python]
糟糕:
"写专业的邮件"

结果:代理发明自己对
"专业"的定义——可能会很死板和企业化。

良好:
"写专业但亲切的邮件。

我们语气的示例:
'你好Sarah,感谢你的联系!我查看了
你的问题,这是我发现的...'

避免这种语气:
'尊敬的客户,我们已收到您的
询问,正在相应处理。'"

结果:代理匹配你的特定声音。
\end{lstlisting}
\end{codebox}

\subsection{错误3:没有边界}

\begin{codebox}
\begin{lstlisting}[style=python]
糟糕:
"回答任何客户问题"

结果:代理可能承诺你没有的功能,
分享机密信息,或编造答案。

良好:
"回答关于产品功能和
一般定价层级的问题。

对于这些主题,不要回答——上报:
- 退款或账单争议 -> support@
- 法律或合规 -> 上报给创始人
- 定制企业定价 -> 收集需求,上报
- 功能承诺 -> '我会和团队确认'

永远不要承诺我们没有的功能。
永远不要分享收入数字。
永远不要讨论其他客户。"

结果:代理知道它的权限在哪里结束。
\end{lstlisting}
\end{codebox}

\subsection{错误4:没有输出格式}

\begin{codebox}
\begin{lstlisting}[style=python]
糟糕:
"评估这个潜在客户"

结果:代理返回非结构化的文字,
难以解析和在自动化中使用。

良好:
"评估这个潜在客户并只返回这个JSON:
{
  'score': 0-100,
  'tier': 'hot' | 'warm' | 'cold',
  'summary': '1-2句总结',
  'next_action': '具体推荐行动',
  'missing_info': ['需要收集的信息列表']
}

不要其他文字。只要JSON。"

结果:输出可以可靠地用于工作流。
\end{lstlisting}
\end{codebox}

\section{组织你的操作手册库}

保持操作手册有序,这样你和你的代理都能找到它们:

\begin{codebox}
\begin{lstlisting}[style=python]
/playbooks
|-- /agents
|   |-- emma-system-prompt.md
|   |-- sam-system-prompt.md
|   |-- maya-system-prompt.md
|   |-- casey-system-prompt.md
|   |-- finn-system-prompt.md
|   `-- oscar-system-prompt.md
|-- /sales
|   |-- lead-qualification.md
|   |-- initial-response.md
|   |-- follow-up-sequence.md
|   |-- objection-handling.md
|   `-- closing-process.md
|-- /marketing
|   |-- blog-post-creation.md
|   |-- social-media.md
|   |-- email-newsletter.md
|   `-- content-repurposing.md
|-- /support
|   |-- ticket-triage.md
|   |-- common-issues.md
|   |-- escalation-matrix.md
|   `-- customer-health.md
|-- /operations
|   |-- order-processing.md
|   |-- inventory-alerts.md
|   `-- vendor-management.md
`-- /templates
    |-- email-responses/
    |-- content-formats/
    `-- reports/
\end{lstlisting}
\end{codebox}

用Git对这些进行版本控制。每个更改都被跟踪。需要时可以回滚。你的操作手册和你的代码一样重要。

\section{测试和迭代}

提示不是一次性编写的。它们随着你的学习而演进。

\subsection{提示测试协议}

\begin{codebox}
\begin{lstlisting}[style=python]
部署前

1. 用10个有代表性的示例测试:
   - 3个典型案例
   - 3个边缘案例
   - 2个困难案例
   - 2个随机/异常案例

2. 检查一致性:
   - 对同一输入运行3次
   - 输出应该基本相似
   - 语气应该一致

3. 验证质量门:
   - 所有输出通过你的质量检查
   - 没有幻觉
   - 格式正确

部署后

每周:
- 随机抽样10%的输出
- 对每个评分 A/B/C/F
- 识别失败模式
- 针对C/F模式更新提示

每月:
- 审查所有上报
- 分析所有边缘案例
- 更新操作手册
- 添加新示例
\end{lstlisting}
\end{codebox}

\begin{keyinsight}[指令公式]
你的代理会完全按照你告诉它们的去做——不多也不少。

\textbf{系统提示}定义身份:代理是谁,它能做什么和不能做什么。

\textbf{操作手册}定义流程:特定任务的分步说明。

\textbf{模板}确保一致性:常见输出的可靠构建块。

\textbf{协议}管理判断:何时行动,何时上报,何时停止。

\textbf{清晰的身份 + 详细的流程 + 好的示例 = 可靠的代理}

还记得我12,000美元的折扣灾难吗?它发生是因为``乐于助人''没有被定义。你的代理会用假设来填补指令中的任何空白。这些假设可能是合理的——或者可能会让你损失12,000美元。

定义一切。不做假设。不断测试。
\end{keyinsight}

\textbf{下一章:}监控你的代理,出问题时调试,以及持续改进性能。
