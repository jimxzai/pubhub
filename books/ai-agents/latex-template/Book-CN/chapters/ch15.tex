\chapter{销售自动化实战手册}

\section{消失的潜在客户}

我仍然记得那种下沉的感觉。那是周五晚上10点,我终于在处理邮件。在十七封邮件的下面,有一封三天前的询问。一位快速增长的金融科技公司的产品副总裁询问演示。他们\textit{``本周正在评估解决方案,想快速推进。''}

那是周二。现在是周五晚上。我没有回复。

我立即发了一封回复——道歉,提供下周的时间。周一早上收到回复:\textit{``谢谢,但我们已经选择了供应商。你们的产品看起来很有趣,但我们需要能快速响应的人。''}

\textbf{一封错过的邮件。一次延迟的回复。一笔至少价值15,000美元的失去的交易。}

这不是一次性的失败。我翻阅了我的收件箱,发现了一个模式:

\begin{itemize}
\item 通过网站表单提交的潜在客户,我好几天都没看到
\item LinkedIn消息淹没在噪音中
\item 推荐介绍埋在长长的邮件线程里
\item 我是自己销售流程的瓶颈
\end{itemize}

就在那时我意识到:问题不是我销售能力差。问题是销售是7x24小时的,而我只是一个需要睡觉、吃饭,偶尔还要做我销售的实际工作的人。

我需要一个永不睡觉的系统。我需要Sam,我的销售代理,运行自动化手册,无论我醒着还是睡着都能捕获、筛选和培育潜在客户。在接下来的六个月里,我建立了一个改变我业务的销售自动化机器。这一章是完整的实战手册。

\section{理解全漏斗自动化}

在我们深入具体自动化之前,你需要了解我们在建造什么:一个完整的销售机器,处理从陌生人到客户再到倡导者的整个旅程——只有在关键决策点才需要你的参与。

这是愿景:

\begin{codebox}
\begin{lstlisting}[style=python]
自动化销售漏斗
--------------------------

认知           -> Maya生成吸引人的内容
     v
兴趣           -> 网站捕获并跟踪访客
     v
考虑           -> Sam筛选和培育
     v
决策           -> Sam预约会议,你成交
     v
购买           -> Finn处理支付
     v
留存           -> Casey确保成功
     v
倡导           -> Maya捕获并分享故事
\end{lstlisting}
\end{codebox}

注意什么是自动化的,什么不是。系统处理认知、初始参与、筛选、培育和会议预约。你出现进行实际的销售对话。然后系统处理其他一切。

当我第一次画出这个图时,我是持怀疑态度的。自动化真的能处理所有这些触点而不感觉像机器人吗?答案是,我发现好的自动化根本不会让人感觉是自动化。它感觉像是一家反应迅速、细心周到的公司,能快速回复相关信息。

\section{建立潜在客户捕获引擎}

第一个要解决的问题很简单:潜在客户实际上是如何到达你这里的,你捕获了所有的吗?

当我审计我的潜在客户来源时,我发现一片混乱:

\begin{itemize}
\item 网站表单提交进入我每天只查看一次的收件箱
\item 聊天小部件对话只有我碰巧在线时才能看到
\item 邮件询问混在垃圾邮件和新闻通讯中
\item LinkedIn私信散落在三个不同的消息类别中
\item 推荐介绍埋在长长的邮件线程里
\end{itemize}

解决方案是统一的潜在客户处理。每个潜在客户,无论来源,都流入同一个筛选管道:

\begin{codebox}
\begin{lstlisting}[style=python]
潜在客户来源 -> 统一处理
---------------------------------

网站表单
|-- Typeform / 原生表单
|-- Webhook到n8n
`-- -> Sam处理

聊天小部件
|-- Intercom / Crisp / Drift
|-- 新潜在客户Webhook
`-- -> Sam处理

邮件询问
|-- contact@company.com
|-- Gmail过滤器 / 转发
`-- -> Sam处理

社交私信
|-- LinkedIn / Twitter
|-- Zapier监控
`-- -> Sam处理

推荐
|-- 客户分享链接
|-- 跟踪参数
`-- -> Sam处理(VIP标记)

全部 -> 同一个筛选管道
\end{lstlisting}
\end{codebox}

这里的关键洞见是整合比复杂更重要。你不需要有五十个集成的花哨CRM。你需要每个潜在客户都进入同一个管道,这样什么都不会漏掉。

我用n8n实现了这个,一个工作流自动化工具。当潜在客户从任何渠道到达时,它触发一个webhook创建统一的潜在客户记录。然后Sam通过筛选管道处理该记录。

\subsection{潜在客户丰富的魔力}

这是自动化变得真正强大的地方。当潜在客户到达时,你通常只有最少的信息:一个名字,一个邮箱,也许一条简短的消息。Sam自动将其转化为丰富的档案:

\begin{codebox}
\begin{lstlisting}[style=python]
自动丰富
--------------------

当潜在客户到达时:

1. 基本信息(来自表单)
   `-- 姓名、邮箱、公司、消息

2. 公司数据(自动丰富)
   |-- 公司规模(LinkedIn/Clearbit)
   |-- 行业
   |-- 位置
   |-- 融资阶段
   `-- 技术栈(如可用)

3. 联系人数据(自动丰富)
   |-- 职位/角色
   |-- LinkedIn档案
   |-- 之前的公司
   `-- 共同联系人

4. 行为数据(跟踪)
   |-- 访问的页面
   |-- 网站停留时间
   |-- 下载的内容
   `-- 邮件打开/点击

5. 意图信号(分析)
   |-- 竞争对手提及
   |-- 痛点关键词
   |-- 预算信号
   `-- 时间表指标
\end{lstlisting}
\end{codebox}

第一次看到这个实际运行时,感觉像魔法。一个潜在客户只提交了他们的名字和邮箱。十秒钟内,Sam已经添加了:他们是一家200人SaaS公司的运营总监,他们最近完成了B轮融资,他们在过去一周访问了我的定价页面四次,他们之前在一家已经是我客户的公司工作过。

这些背景改变了对话。不是通用的"感谢你的兴趣,我能帮你什么?"我可以写(或Sam可以起草):"嗨Sarah——我看到你在[公司]。有趣的是你来自[之前的公司];他们已经是我们两年的客户了。根据你查看的页面,看起来你在评估[具体用例]的选项。想快速通话讨论一下吗?"

丰富管道很简单:

\begin{codebox}
\begin{lstlisting}[style=python]
丰富管道
-------------------

触发:新潜在客户到达

步骤1:提取域名
`-- john@techcorp.com -> techcorp.com

步骤2:公司查询
`-- Clearbit / LinkedIn API
    返回:规模、行业、融资

步骤3:联系人查询
`-- LinkedIn档案搜索
    返回:职位、历史

步骤4:CRM检查
`-- 现有客户?之前的潜在客户?
    返回:历史、关系

步骤5:组合上下文
`-- 创建丰富的潜在客户档案
    传递给Sam进行筛选

时间:总共5-10秒
成本:每次丰富约$0.10
\end{lstlisting}
\end{codebox}

每个潜在客户约十美分,你获得的背景信息人类需要二十分钟才能研究出来。而且它是即时发生的,允许立即个性化跟进。

\section{设计多触点序列}

一旦潜在客户被捕获和丰富,下一个挑战是培育他们走向对话。这是大多数独立运营者失败的地方。他们要么回复一次然后忘记,要么跟进得太激进以至于惹恼了潜在客户。

我开发了两个核心序列:一个用于主动联系我的温和潜在客户,一个用于不认识我的潜在客户的冷外联。

\subsection{温和培育序列}

当有人通过我的网站联系或回应内容时,他们是温和的。他们已经表达了兴趣。目标是高效地推动他们走向会议而不咄咄逼人:

\begin{codebox}
\begin{lstlisting}[style=python]
序列:温和潜在客户培育
--------------------------

目标:将温和潜在客户推向会议
持续时间:14天
触点:5个

第0天:初始回复
|-- 基于询问个性化
|-- 1个资格问题
`-- 软性行动号召:"回复你的想法"

第3天:价值输送
|-- 相关内容
|-- 基于他们的行业/角色
`-- 无要求,纯价值

第7天:案例研究
|-- 类似公司成功故事
|-- 具体指标
`-- 软性行动号召:"想看看怎么做到的?"

第10天:直接邀请
|-- 承认日程繁忙
|-- 提供具体时间
`-- 明确行动号召:会议

第14天:告别
|-- 最后一次尝试
|-- 留下机会
`-- 提供替代方案(新闻通讯)

退出条件:
|-- 回复 -> Sam接管
|-- 会议预约 -> 结束序列
|-- 退订 -> 结束序列
|-- 硬退信 -> 移除
\end{lstlisting}
\end{codebox}

这个序列的关键是节奏。第0天是即时和响应的。第3天证明你有价值可以提供。第7天展示社会证明。第10天是直接邀请。第14天是优雅的退出。

注意退出条件。潜在客户一旦回复,他们就退出自动序列,Sam用动态的、对话式的跟进接管。自动化处理可预测的;智能处理不可预测的。

\subsection{冷外联序列}

冷外联更难。这些潜在客户不认识你,也没有要求收到你的消息。你需要赢得他们的注意:

\begin{codebox}
\begin{lstlisting}[style=python]
序列:冷外联
-----------------------

目标:从冷名单生成兴趣
持续时间:21天
触点:4个

第0天:打破模式
|-- 意想不到的主题行
|-- 关于他们公司的观察
|-- 问题,而非推销
`-- 示例:"关于{{recent_news}}的快速问题"

第5天:社会证明
|-- 像他们这样的公司
|-- 具体结果
|-- "想你可能会感兴趣"
`-- 无硬性要求

第12天:直接价值
|-- 这是我们做什么
|-- 这是我们解决的问题
|-- 这是结果
`-- "值得聊聊吗?"

第21天:请求许可
|-- "看起来时机不对"
|-- "我应该3个月后再联系吗?"
|-- 或者"想现在聊聊吗?"
`-- 二选一

个性化:
|-- {{company_news}} - 近期公告
|-- {{industry_trend}} - 相关趋势
|-- {{competitor}} - 如果使用竞争对手
|-- {{mutual_connection}} - 如果有
\end{lstlisting}
\end{codebox}

冷序列触点更少,间隔更长,个性化更多。每封邮件都需要证明你对这个特定潜在客户做了功课。

第21天的"请求许可"至关重要。你承认他们没有回复,同时给他们一个简单的方式要么参与要么明确说"现在不行"。我很多最好的交易来自回复这最后一封邮件的潜在客户:"实际上,下个季度会更好。三月联系我。"

\subsection{序列逻辑和路由}

不是每个潜在客户都应该得到同样的序列。Sam根据他们的特征路由潜在客户:

\begin{codebox}
\begin{lstlisting}[style=python]
序列决策树
----------------------

潜在客户分数?
|-- 80+(热门)-> 立即回复,无序列
|-- 50-79(温和)-> 温和序列
|-- 25-49(培育)-> 冷序列
`-- <25(冷淡)-> 仅新闻通讯

序列期间:
|-- 打开邮件 -> 继续
|-- 点击链接 -> 跳到下一步
|-- 回复 -> 退出,Sam接管
|-- 2个后无参与 -> 放慢
`-- 退信 -> 移除

序列后:
|-- 有参与但无会议 -> 每月新闻通讯
|-- 无参与 -> 每季度检查
`-- 负面回应 -> 从外联中移除
\end{lstlisting}
\end{codebox}

热门潜在客户完全跳过序列——Sam立即回复会议请求。冷淡潜在客户得到更长、更柔和的序列。有参与但未转化的潜在客户移入长期培育而不是被放弃。

\section{自动化会议预约}

当潜在客户准备好交谈时,你最不想要的就是摩擦。我因为对日程安排请求回复太慢而失去会议。我因为我们打了三天邮件乒乓试图找到合适的时间而失去会议。

Sam用日历智能处理这个:

\begin{codebox}
\begin{lstlisting}[style=python]
智能日程安排
----------------

Sam读取你的日历并提供:

可用性规则:
|-- 周一至周四:上午9点 - 下午4点(你的时区)
|-- 周五:上午9点 - 中午12点(短日)
|-- 缓冲:会议前后15分钟
|-- 不连续通话
`-- 每天最多4个外部通话

优先预约:
|-- 热门潜在客户:今天/明天时段
|-- 温和潜在客户:本周时段
|-- 演示:首选上午时段
|-- 跟进:首选下午时段
`-- 企业:根据他们的需要灵活

智能建议:
|-- "根据你的消息,30分钟演示
|    比较合适。周二下午2点怎么样?"
|-- 如果拒绝:"没问题!这里有
|    其他选项:周三上午10点,周四下午3点"
`-- 如果时区不清楚:"你在哪个时区
     ?我来调整。"
\end{lstlisting}
\end{codebox}

可用性规则至关重要。在我实施这些之前,我会把会议安排得背靠背,没有时间准备,然后精疲力竭。现在Sam强制执行我自己从来没有足够自律去执行的边界。

\subsection{会前准备}

Sam预约的每个会议都会触发准备管道:

\begin{codebox}
\begin{lstlisting}[style=python]
会议准备管道
---------------------

触发:会议预约于T+24小时

T-24小时:确认
|-- 发送日历邀请
|-- 包含议程
|-- 分享准备材料
`-- 问:"有什么具体要讨论的吗?"

T-4小时:准备包(给你)
|-- 拉取所有潜在客户上下文
|-- 公司研究摘要
|-- 类似客户案例
|-- 建议谈话要点
|-- 潜在异议

T-1小时:提醒
|-- 给潜在客户发邮件
|-- "期待我们的通话!"
|-- 包含会议链接
`-- 你的电话作为备用

T+0:会议进行

T+1小时:跟进草稿
|-- Sam起草感谢信
|-- 包含讨论要点
|-- 明确下一步
`-- 准备好供你审阅/发送
\end{lstlisting}
\end{codebox}

T-4小时的准备包彻底改变了我的会议。我过去盲目地进入通话,问一些我花五分钟研究就能回答的基本问题。现在我带着完整的简报走进去:他们是谁,他们关心什么,我们帮助过哪些类似客户,可能会出现什么异议。

\subsection{处理爽约}

即使有确认邮件和提醒,一些潜在客户也不会出现。为此有一个自动化流程消除了在被放鸽子后跟进的情感摩擦:

\begin{codebox}
\begin{lstlisting}[style=python]
爽约序列
----------------

T+5分钟:等待
`-- 等待,他们可能迟到

T+10分钟:第一次联系
`-- 邮件:"只是确认你收到链接了!"
    再次包含会议URL

T+15分钟:结束会议
`-- 离开,他们不会来了

T+30分钟:重新安排提议
`-- "抱歉我们错过了!
     这里有一些其他时间..."

T+24小时:最后一次尝试
`-- "还有兴趣吗?很乐意重新安排
     或异步回答问题。"

T+72小时:如果无回复
`-- 移入培育序列
\end{lstlisting}
\end{codebox}

自动化处理了本来会是尴尬的、经常被忽略的任务。我不必决定是否跟进——系统为我做了,专业且没有怨恨。

\section{管理交易管道}

随着交易推进,它们会经过不同阶段。每个阶段都有特定的自动化来保持事情推进:

\begin{codebox}
\begin{lstlisting}[style=python]
管道阶段
---------------

阶段1:新潜在客户
|-- 自动:丰富数据
|-- 自动:评分和筛选
|-- 自动:适当路由
`-- 退出:移至已筛选或培育

阶段2:已筛选
|-- 自动:添加到响应序列
|-- 自动:如果热门则预约会议
|-- 监控:4小时内响应
`-- 退出:移至已安排会议

阶段3:已安排会议
|-- 自动:发送确认
|-- 自动:T-4小时准备包
|-- 自动:T-1小时提醒
`-- 退出:会议进行 -> 发现

阶段4:发现
|-- 自动:2小时内跟进
|-- 自动:创建提案任务
|-- 监控:定义下一步
`-- 退出:移至提案

阶段5:提案
|-- 自动:跟踪提案打开
|-- 自动:48小时未打开则跟进
|-- 监控:问题/异议
`-- 退出:移至谈判或失去

阶段6:谈判
|-- 警报:需要创始人关注
|-- 跟踪:决策时间表
|-- 自动:处理常规问题
`-- 退出:移至赢得或失去

阶段7:赢得
|-- 自动:Slack庆祝消息!
|-- 自动:触发Finn(开票)
|-- 自动:触发Casey(入职)
`-- 退出:交接完成

阶段8:失去
|-- 自动:记录原因
|-- 自动:移入长期培育
|-- 自动:安排6个月后检查
`-- 学习:我们为什么失去?
\end{lstlisting}
\end{codebox}

基于阶段的自动化的力量是什么都不会被遗忘。每个合格的潜在客户都会得到及时响应。每个发现通话都会被跟进。每个提案都会被跟踪。每个失去的交易都会被记录和分析。

\subsection{检测卡住的交易}

交易会停滞。这是不可避免的。问题是你是否在为时已晚之前注意到。Sam监控交易速度,当有问题时发出警报:

\begin{codebox}
\begin{lstlisting}[style=python]
交易速度警报
--------------------

卡住交易检测:

阶段:已安排会议
|-- 正常:2-7天
|-- 警报如果:>7天没有会议
`-- 行动:检查是否需要重新安排

阶段:发现
|-- 正常:1-3天跟进
|-- 警报如果:>3天没有下一步
`-- 行动:发送跟进

阶段:提案
|-- 正常:3-7天做决定
|-- 警报如果:>10天没有进展
`-- 行动:检查通话

阶段:谈判
|-- 正常:1-2周
|-- 警报如果:>3周
`-- 行动:强制决定或取消资格

速度仪表板:
|-- 每个阶段的平均天数
|-- 阶段间转化率
|-- 瓶颈识别
`-- 随时间趋势
\end{lstlisting}
\end{codebox}

在我有这些警报之前,交易会悄悄死去。一个潜在客户在收到提案后会沉默,三周后我碰巧审查管道时才注意到。到那时,他们通常已经从别人那里买了。

现在交易一偏离预期速度我就会收到警报。一个在提案阶段卡了十天的交易会触发Sam的自动检查。这通常会浮现我需要的信息:"抱歉,太忙了。我们下周能谈谈吗?"或"我们需要财务批准,三天后回复你。"

\section{系统化处理异议}

每个销售人员都面临同样的异议:价格、时机、权限、竞争。成功和失败的区别在于你如何处理它们。Sam有一个完整的异议处理手册:

\begin{codebox}
\begin{lstlisting}[style=bash]
# 异议处理手册

## 价格异议

### "太贵了"
响应方法:
1. 承认担忧
2. 重新定义为投资
3. 展示ROI计算
4. 如需要提供替代方案

示例:
"我理解——确保投资有意义很重要。
让我展示像你这样的客户
通常看到的回报...

类似规模的[公司X]看到:
- 每周节省10小时 = 每月$2,000价值
- 每月多2笔交易 = 每月$4,000价值
- 总价值:$6,000/月,花费$500/月

对你的情况做类似计算有帮助吗?"

### "存在更便宜的替代品"
响应:
"你说得对,有更便宜的选项。问题是
实际成本是什么?从[更便宜选项]切换到我们的
客户告诉我们他们每月花15+小时
在变通方法上。按你们团队的费率,那是每月$X的
隐藏成本。想让我给你看比较吗?"
\end{lstlisting}
\end{codebox}

异议处理不止于此。Sam还处理时机和权限异议:

\begin{codebox}
\begin{lstlisting}[style=bash]
## 时机异议

### "不是合适的时机"
响应:
"完全理解——时机很重要。快速问一下:不是
合适的时机是因为预算周期,还是因为
其他优先级在这之前?

[如果预算]:你们下一个周期什么时候开始?我会
到时带着更新的信息跟进。

[如果优先级]:什么需要改变才能让这个
成为优先级?有时候我可以帮忙说服。"

## 权限异议

### "需要和老板确认"
响应:
"当然——重要决定应该让合适的
人参与。如果我准备一页总结
给他们一切有帮助吗?我可以包含我们讨论的
ROI计算。

如果需要的话也很乐意和他们
快速通话。"
\end{lstlisting}
\end{codebox}

\subsection{自动异议检测}

Sam不仅准备好了响应——他还主动在潜在客户沟通中检测异议:

\begin{codebox}
\begin{lstlisting}[style=python]
异议检测
-------------------

Sam分析响应中的异议信号:

价格信号:
|-- "贵" / "成本" / "预算"
|-- "更便宜的替代品"
|-- "需要证明花费合理"
`-- -> 路由到价格处理

时机信号:
|-- "现在不行" / "以后" / "下季度"
|-- "太忙" / "优先级"
|-- "X时间后再联系"
`-- -> 路由到时机处理

权限信号:
|-- "需要和...确认" / "老板"
|-- "委员会" / "团队决定"
|-- "不是我能决定的"
`-- -> 路由到权限处理

竞争信号:
|-- 提到竞争对手名称
|-- "在看其他选项"
|-- "已经在用X"
`-- -> 路由到竞争处理

行动:标记异议类型,建议响应模板
\end{lstlisting}
\end{codebox}

当潜在客户提到"预算"或"需要和我的团队确认"时,Sam会标记它并建议适当的响应框架。这种一致性意味着每个异议都得到很好的处理,而不仅仅是我碰巧在那一刻想到的。

\section{竞争情报自动化}

了解你的竞争对手不是一次性的研究项目。它是一个持续的情报行动。Sam自动监控竞争对手:

\begin{codebox}
\begin{lstlisting}[style=python]
竞争跟踪
--------------------

自动监控:

新闻和公告:
|-- 竞争对手名称的Google Alerts
|-- Crunchbase融资
|-- LinkedIn招聘模式
`-- -> 每周摘要给创始人

定价变化:
|-- 每月检查定价页面
|-- 用Wayback Machine存档
|-- 比较并在变化时警报
`-- -> 重大变化立即警报

功能发布:
|-- 监控更新日志/博客
|-- 跟踪Product Hunt
|-- Twitter提及
`-- -> 大功能立即警报

评论网站:
|-- G2、Capterra评论
|-- 情感分析
|-- 常见投诉
`-- -> 季度总结
\end{lstlisting}
\end{codebox}

这种情报输入竞争定位。当潜在客户提到竞争对手时,Sam可以调出我们的定位框架:

\begin{codebox}
\begin{lstlisting}[style=bash]
## vs. 竞争对手A(市场领导者)

他们的优势:
- 品牌认知度
- 功能完整性
- 大客户基础

他们的劣势:
- 昂贵
- 实施复杂
- 支持响应慢

我们的定位:
"[竞争对手A]对于有专门团队的企业很好。
如果你是一家需要快速行动而不需要
大型实施项目的成长型公司,我们就是为你打造的。"

关键证明点:
- 实施:2天 vs 2个月
- 成本:低70%
- 支持:2小时响应 vs 2天
\end{lstlisting}
\end{codebox}

每次提到竞争对手的销售对话现在都会自动触发正确的定位。我永远不会被"为什么我们应该用你而不是[大公司]?"问倒。

\section{真正有用的销售报告}

大多数销售报告是信息坟墓——没人看的数字。我建立的报告驱动行动。

\subsection{每日销售摘要}

每天早上7点,Sam给我发一个简短的摘要:

\begin{codebox}
\begin{lstlisting}[style=python]
每日销售报告
------------------

发送:每日早上7点

昨日活动
|-- 新潜在客户:8
|-- 已筛选:5(62.5%)
|-- 预约会议:2
|-- 发送提案:1
|-- 赢得交易:0
`-- 失去交易:0

管道状态
|-- 活跃机会:23
|-- 价值:$85,000
|-- 加权价值:$34,000
|-- 预计本月成交:$22,000
`-- 有风险:2笔交易(卡住>7天)

今日优先级
|-- 跟进:TechCorp提案(3天前发送)
|-- 会议:Acme Corp演示下午2点
|-- 准备:DataFlow发现下午4点
`-- 审查:3个热门潜在客户等待响应

警报
|-- [!] TechCorp没有打开提案
`-- [!] DataFlow负责人下周休假
\end{lstlisting}
\end{codebox}

这个摘要只需要30秒阅读,准确告诉我今天要关注什么。它不仅仅是信息——它是优先排序的行动项目。

\subsection{每周管道审查}

每周一,我得到更深入的分析:

\begin{codebox}
\begin{lstlisting}[style=python]
每周管道审查
----------------------

漏斗指标
                   本周    上周    变化
---------------------------------------------------
新潜在客户            42       38       +10%
-> 已筛选             28       25       +12%
-> 会议               12       10       +20%
-> 提案                5        6       -17%
-> 赢得                2        1      +100%

转化率
|-- 潜在客户 -> 已筛选:  67%(目标:60%)
|-- 已筛选 -> 会议:      43%(目标:40%)
|-- 会议 -> 提案:        42%(目标:50%)[!]
|-- 提案 -> 赢得:        40%(目标:30%)
`-- 整体 L->W:          4.8%(目标:5%)

顶级机会
1. TechCorp - $15,000 - 70% - 提案阶段
2. Acme Corp - $12,000 - 50% - 发现
3. DataFlow - $8,000 - 60% - 已安排会议

本周失去
1. StartupX - $5,000 - 原因:选择了竞争对手
   学习:需要更好的竞争定位

有风险
1. MidCo - $10,000 - 45% - 10天无响应
   行动:周一电话

下周重点
1. 成交TechCorp(推动决定)
2. 将Acme会议转化为提案
3. 重新联系MidCo
\end{lstlisting}
\end{codebox}

每周审查展示瓶颈在哪里。这周,我的会议到提案转化低于目标。这告诉我需要改进我的发现通话——我在进行好的对话但没有有效推进它们。

\section{连接代理}

销售不是孤立存在的。Sam与Maya(营销)和Casey(客户成功)合作创建无缝交接。

\subsection{营销到销售交接}

当Maya的内容产生参与时,高意图访客成为Sam的潜在客户:

\begin{codebox}
\begin{lstlisting}[style=python]
营销 -> 销售交接
-------------------------

Maya的内容 -> Sam的管道:

1. 博客文章发布
   `-- Maya跟踪参与

2. 检测到高意图行为:
   |-- 下载门控内容
   |-- 访问定价3+次
   |-- 一次会话多篇博客文章
   `-- -> 为Sam创建潜在客户

3. Sam收到带上下文的:
   |-- 他们参与的内容
   |-- 感兴趣的话题
   `-- 个性化外联角度
\end{lstlisting}
\end{codebox}

当访客下载我们的"企业集成指南"然后访问定价页面三次,Maya为Sam创建一个带上下文的潜在客户:"这个人对企业集成感兴趣,正在积极评估定价。"

\subsection{销售到成功交接}

当Sam成交一笔交易,Casey需要上下文确保顺利入职:

\begin{codebox}
\begin{lstlisting}[style=python]
销售 -> 成功交接
-----------------------

当交易成交时:

Sam传递给Casey:
|-- 完整对话历史
|-- 讨论的痛点
|-- 他们兴奋的功能
|-- 实施预期
|-- 关键利益相关者
|-- 他们提到的成功指标
`-- 做出的任何承诺

Casey收到并:
|-- 创建客户档案
|-- 安排启动会
|-- 准备入职计划
|-- 设置健康评分基线
`-- 开始成功跟踪
\end{lstlisting}
\end{codebox}

交接中不会丢失任何信息。Casey确切知道客户被承诺了什么,他们对什么感兴趣,以及成功对他们来说是什么样子。

\section{转型}

还记得本章开头我失去的那个潜在客户吗?那个因为我响应不够快而从别人那里买的产品副总裁?

实施这些自动化六个月后,我收到了来自那家公司的另一个询问。不同的人,同一家公司,同样的需求。这次,Sam在两分钟内响应——在周六——根据他们的行业和角色发送个性化消息。到周一早上,他们已经安排了周二的会议。

我们成交了那笔交易。18,000美元年度合同。原来的潜在客户甚至加入了对话:"我很高兴你们变快了。我们对另一家供应商不满意。"

这就是销售自动化创造的转型。不仅仅是效率,而是你本来会失去的收入。你本来会搞砸的关系。你本来会忘记的交易。

\begin{keyinsight}{销售自动化公式}
有效的销售自动化遵循80/10/10规则:自动化80\%的活动(捕获、丰富、培育、调度、报告),对10\%发出警报(卡住的交易、异议、竞争提及),将你的人类注意力集中在剩余的10\%(实际销售对话、复杂谈判、关系建设)。Sam处理机器工作,这样你就可以专注于只有你能做的人类工作。
\end{keyinsight}

\vspace{1em}
\textbf{下一章:}销售产生收入后,你如何扩展营销来填充管道?Maya的营销自动化系统创建内容、分发内容、跟踪有效的——所有这些都在你睡觉时进行。
