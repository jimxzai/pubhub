\chapter{Building Your Agent Stack}

\section{The Stack That Almost Killed My Business}

I remember the night my automation infrastructure collapsed.

It was 2 AM, and I woke to seventeen angry customer emails, a Slack channel on fire, and a payment system that had been down for six hours. The culprit? A cascade of failures:

\begin{itemize}
\item My workflow tool had an unexpected outage
\item Which triggered errors in my AI integration
\item Which corrupted the state of my customer database
\item Which sent the same invoice to one customer forty-seven times
\end{itemize}

I'd built my agent stack like a Rube Goldberg machine. Every piece depended on every other piece. When one domino fell, they all fell.

The next morning, exhausted and embarrassed, I started over. This time, I built for resilience. For simplicity. For the kind of reliability that lets you sleep at night.

That stack is what I'm going to teach you to build in this chapter.

\section{The Build vs. Buy Decision}

Before we dive into the technical architecture, let's address the fundamental question every founder faces: Should you use off-the-shelf AI tools or build custom agents?

The honest answer is both, in sequence.

Start with existing tools. They'll cover eighty percent of your needs with ten percent of the effort. Only build custom when you hit their limits---and you'll know when you hit them because the workarounds become more complex than building from scratch.

\begin{codebox}
\begin{lstlisting}[style=python]
BUILD CUSTOM WHEN:
- Your process is genuinely unique
- Integration costs exceed build costs
- You need complete control over behavior
- Data privacy is non-negotiable
- You have technical capacity (or budget for it)

USE EXISTING TOOLS WHEN:
- Standard workflows (email, calendar, CRM)
- Speed to market matters more than perfection
- You're still figuring out the process
- Budget is limited
- No technical team available

THE PATTERN: Start with existing tools, identify
gaps, build custom only for those gaps.
\end{lstlisting}
\end{codebox}

I made the mistake of building custom too early. I spent three weeks creating a bespoke email triage system when a combination of Claude Pro and Zapier would have handled ninety percent of the work. Those three weeks could have been spent acquiring customers.

Build when you must. Buy when you can. Your competitive advantage isn't your infrastructure---it's what you do with it.

\section{The Three-Tier Architecture}

I've come to think of agent infrastructure in three tiers, each building on the last. Most solo founders should start at Tier 1 and only move up when they outgrow it.

\subsection{Tier 1: No-Code (Where Everyone Should Start)}

This is where you validate that AI agents actually work for your business. The investment is minimal. The learning is maximal.

\begin{codebox}
\begin{lstlisting}[style=python]
THE NO-CODE STACK

Chat Interfaces (Pick one):
|-- Claude.ai Pro ($20/mo)
|   `-- Best for: Complex reasoning, long documents
|-- ChatGPT Plus ($20/mo)
|   `-- Best for: Custom GPTs, plugin ecosystem
`-- Gemini Advanced ($20/mo)
    `-- Best for: Google Workspace integration

Workflow Automation (Pick one):
|-- Zapier ($20-50/mo)
|   `-- Best for: Simple if-this-then-that
|-- Make ($9-16/mo)
|   `-- Best for: Complex multi-step workflows
`-- n8n ($0 self-hosted, $20/mo cloud)
    `-- Best for: AI-native, developer-friendly

AI-Native Tools (Add as needed):
|-- Lindy ($49/mo)
|   `-- Best for: Email, calendar, research
|-- Bardeen ($10/mo)
|   `-- Best for: Browser automation
`-- Relay.app ($50/mo)
    `-- Best for: Human-in-the-loop workflows

TOTAL INVESTMENT: $100-200/month
CAPABILITY: 80% of what you need
TIME TO DEPLOY: Days, not weeks
\end{lstlisting}
\end{codebox}

Here's what this stack looks like in practice. Every morning, Claude Pro reads my emails (via Make integration), categorizes them, drafts responses, and puts them in a review queue. Zapier handles the mechanical work---forwarding, filing, scheduling. I spend fifteen minutes reviewing instead of two hours drowning.

\subsection{Tier 2: Low-Code (When You Outgrow Tier 1)}

You'll know you've outgrown Tier 1 when you start hitting limits: the no-code tools can't handle your edge cases, the costs of multiple subscriptions exceed what building would cost, or you need integrations that don't exist.

\begin{codebox}
\begin{lstlisting}[style=python]
THE LOW-CODE STACK

Visual Agent Builders:
|-- Flowise (open source, free)
|   `-- Visual LangChain builder, self-hosted
|-- Dify.ai (free tier available)
|   `-- Visual agent workflows, managed
`-- Voiceflow ($50/mo)
    `-- Voice and chat agent builder

Database + AI:
|-- Supabase + pgvector (generous free tier)
|   `-- PostgreSQL with vector search
|-- Pinecone ($0-70/mo)
|   `-- Purpose-built vector database
`-- Weaviate (open source)
    `-- Self-hosted vector search

Custom Integrations:
|-- Pipedream (free-$19/mo)
|   `-- Code-level workflow builder
|-- Retool ($10/mo)
|   `-- Internal tools with AI
`-- Airplane (free tier)
    `-- Task and workflow runner

TOTAL INVESTMENT: $50-200/month + time
CAPABILITY: 95% of what you need
TIME TO DEPLOY: Weeks
\end{lstlisting}
\end{codebox}

At this tier, you're building real systems. Your customer knowledge base lives in a vector database. Your agents query it for context. Your workflows handle complex branching logic.

\subsection{Tier 3: Code-First (Maximum Control)}

This is where you go when AI is core to your product, when you need complete control, or when you're operating at scale where custom optimization matters.

\begin{codebox}
\begin{lstlisting}[style=python]
THE CODE-FIRST STACK

AI SDKs:
|-- Anthropic SDK (Python/TypeScript)
|   `-- Direct Claude API access
|-- OpenAI SDK (Python/TypeScript)
|   `-- GPT models and assistants
`-- LangChain / LlamaIndex
    `-- Agent frameworks with tools

Infrastructure:
|-- Modal ($0-30/mo)
|   `-- Serverless Python execution
|-- Railway ($5/mo)
|   `-- Simple app deployment
`-- Fly.io ($0-20/mo)
    `-- Global edge deployment

Observability:
|-- LangSmith (free tier)
|   `-- LangChain tracing and debugging
|-- Helicone (free tier)
|   `-- LLM observability and analytics
`-- Weights & Biases
    `-- ML experiment tracking

Orchestration:
|-- Temporal (open source)
|   `-- Workflow orchestration
|-- Prefect (free tier)
|   `-- Data pipeline orchestration
`-- Dagster (free tier)
    `-- Asset-based orchestration

TOTAL INVESTMENT: $50-200/month + development
CAPABILITY: Everything you can imagine
TIME TO DEPLOY: Months of development
\end{lstlisting}
\end{codebox}

Most solo founders never need Tier 3. But if you do, you'll know---because Tier 2 constraints will feel like handcuffs.

\section{Stacks by Business Type}

Let me give you specific recommendations based on common business models.

\subsection{E-Commerce / Product Business}

Selling physical products means managing orders, inventory, shipping, and customer service. Here's the stack that works:

\begin{codebox}
\begin{lstlisting}[style=python]
E-COMMERCE STACK

Core Platform:
|-- Shopify (your store)
|-- Claude Pro (agent brain)
|-- n8n (orchestration)
`-- Notion (knowledge base)

Agent Configuration:
|-- Oscar: Order processing
|   Flow: Shopify -> n8n -> Claude -> Shipping
|-- Casey: Customer support
|   Flow: Email -> Claude -> Response/Escalate
|-- Finn: Invoicing and reconciliation
|   Flow: Orders -> n8n -> Stripe -> Notifications
`-- Maya: Product descriptions
    Flow: Photos -> Claude -> Shopify listings

Integration Points:
|-- Shopify webhooks trigger n8n workflows
|-- Email (Gmail API) connects to Claude
|-- Shipping (EasyPost/ShipStation) automated
`-- Payments (Stripe) fully automated

Monthly Cost: $200-300 total
\end{lstlisting}
\end{codebox}

The key insight for e-commerce: automate the repetitive, escalate the exceptions. Ninety percent of orders are straightforward. Your agents handle those. The ten percent that need attention get routed to you.

\subsection{Service Business / Agency}

Service businesses run on relationships and deliverables. The stack focuses on client communication and project management.

\begin{codebox}
\begin{lstlisting}[style=python]
SERVICE BUSINESS STACK

Core Platform:
|-- Cal.com (scheduling)
|-- Claude Pro (agent brain)
|-- Make or n8n (orchestration)
`-- Obsidian (client knowledge)

Agent Configuration:
|-- Emma: Email and calendar management
|   Flow: Gmail -> Claude -> Response/Schedule
|-- Sam: Lead qualification and intake
|   Flow: Form -> Claude -> CRM -> Follow-up
|-- Casey: Client success and check-ins
|   Flow: Schedule -> Claude -> Outreach
`-- Finn: Proposals and invoicing
    Flow: Scope -> Claude -> Proposal -> Invoice

Integration Points:
|-- Cal.com syncs with Google Calendar
|-- Forms (Typeform/Tally) feed into CRM
|-- Stripe handles payment collection
`-- Slack notifications for urgent items

Monthly Cost: $150-250 total
\end{lstlisting}
\end{codebox}

For service businesses, the critical automation is the space between client communications. Every email gets a timely response. Every check-in happens on schedule. Nothing falls through the cracks.

\subsection{SaaS / Software Product}

If you're building software, your agents can integrate directly with your product.

\begin{codebox}
\begin{lstlisting}[style=python]
SAAS STACK

Core Platform:
|-- Your application (the product)
|-- Claude API (agent brain)
|-- Supabase (database and auth)
`-- Vercel (deployment)

Agent Configuration:
|-- Sam: Trial to paid conversion
|   Flow: Usage signals -> Outreach
|-- Casey: Onboarding and support
|   Flow: Intercom -> Claude -> Response
|-- Maya: Content and SEO
|   Flow: Claude -> Blog -> Social
`-- Oscar: DevOps and monitoring
    Flow: Alerts -> Triage -> Escalate

Integration Points:
|-- Product events stream to Supabase
|-- Supabase triggers Claude analysis
|-- Intercom handles support automation
`-- GitHub connects to deployment

Monthly Cost: $100-200 (plus API usage)
\end{lstlisting}
\end{codebox}

SaaS businesses have an advantage: your users are already in your system. You know their behavior. Your agents can be proactive based on usage patterns rather than reactive to emails.

\section{Core Integration Patterns}

Regardless of which stack you choose, your agents will follow a few fundamental patterns. Master these and you can build almost anything.

\subsection{Pattern 1: Event $\rightarrow$ AI $\rightarrow$ Action}

This is the most common pattern. Something happens, AI processes it, action is taken.

\begin{codebox}
\begin{lstlisting}[style=python]
TRIGGER -> PROCESS -> ACT

Example: New Lead Processing

1. TRIGGER
   Typeform webhook fires with new submission
   {name: "Sarah", email: "sarah@techcorp.com",
    company: "TechCorp", message: "Need help with..."}

2. PROCESS (n8n + Claude)
   - Enrich with LinkedIn data
   - Score against BANT criteria
   - Result: 85/100 (Hot lead)

3. ACT
   - Create CRM record
   - Send personalized response
   - Schedule follow-up task
   - Notify you via Slack

Time: Under 60 seconds from submission to response
\end{lstlisting}
\end{codebox}

\subsection{Pattern 2: Schedule $\rightarrow$ Gather $\rightarrow$ Synthesize}

This pattern powers all your reporting and monitoring. On a schedule, gather data, have AI synthesize insights.

\begin{codebox}
\begin{lstlisting}[style=python]
SCHEDULE -> GATHER -> SYNTHESIZE

Example: Daily Business Summary

1. SCHEDULE
   6 AM daily trigger

2. GATHER
   - Pull yesterday's sales from Stripe
   - Check support ticket status
   - Get marketing metrics from analytics
   - Review cash position

3. SYNTHESIZE (Claude)
   Generate executive summary with:
   - Key metrics and comparisons
   - Alerts and concerns
   - Recommendations for today

4. DELIVER
   Email summary to your inbox before you wake up
\end{lstlisting}
\end{codebox}

\subsection{Pattern 3: Monitor $\rightarrow$ Detect $\rightarrow$ Respond}

This pattern watches for problems and responds before they become crises.

\begin{codebox}
\begin{lstlisting}[style=python]
WATCH -> IDENTIFY -> ACT

Example: Customer Health Monitoring

1. WATCH
   Every 4 hours, pull usage metrics:
   - Login frequency
   - Feature usage
   - Support ticket patterns

2. IDENTIFY (Claude)
   Analyze for warning signs:
   "Customer usage dropped 50% in 7 days"
   "No login in 14 days"
   "Negative sentiment in recent ticket"

3. ACT
   - Send proactive check-in email
   - Create internal alert
   - If no response in 48h, escalate
   - If health critical, notify founder immediately
\end{lstlisting}
\end{codebox}

\subsection{Pattern 4: Request $\rightarrow$ Enrich $\rightarrow$ Complete}

This pattern powers content creation and research tasks. Take an input, enrich with context, produce output.

\begin{codebox}
\begin{lstlisting}[style=python]
INPUT -> AUGMENT -> OUTPUT

Example: Content Creation

1. INPUT
   "Write a blog post about customer churn"

2. AUGMENT
   - Search knowledge base for relevant notes
   - Pull recent customer stories and examples
   - Get current statistics from industry sources
   - Load brand voice guide

3. COMPLETE (Claude with context)
   Generate 1500-word blog post that:
   - Matches your voice
   - Includes real examples
   - Cites accurate data

4. OUTPUT
   - Save draft to content folder
   - Generate 3 social snippets
   - Queue for human review
\end{lstlisting}
\end{codebox}

\section{Why I Recommend n8n}

After trying Zapier, Make, Pipedream, and custom code, I settled on n8n as my orchestration layer. Here's why:

\begin{codebox}
\begin{lstlisting}[style=python]
N8N ADVANTAGES

Open Source:
- Self-host for free (or $20/mo cloud)
- No vendor lock-in
- Full control over your data
- Inspect and modify anything

AI-Native:
- Built-in nodes for Claude, OpenAI, etc.
- Vector database support
- Code nodes for custom logic
- Designed for AI workflows

Integration Rich:
- 400+ pre-built integrations
- HTTP node for anything else
- Webhooks in and out
- Direct database connections

Developer Friendly:
- JavaScript/Python in code nodes
- Git-based version control possible
- Environment variables
- Self-documenting workflows
\end{lstlisting}
\end{codebox}

Let me show you a real workflow. This is how I process leads:

\begin{codebox}
\begin{lstlisting}[style=python]
N8N WORKFLOW: Lead Processor

Node 1: [Webhook]
- Receives POST /webhook/new-lead
- Body: {name, email, company, message}

Node 2: [HTTP Request]
- Enriches with company data from Clearbit
- Adds: {size, industry, funding}

Node 3: [Claude]
- System: "You are a lead qualification expert..."
- Prompt: "Score this lead 0-100 based on..."
- Output: {score: 85, tier: "hot", summary: "..."}

Node 4: [IF]
- If score >= 80: Hot path
- If score >= 50: Warm path
- Else: Cold path

Node 5a (Hot): [Parallel]
- Create Notion record
- Send immediate response
- Slack notification
- Schedule follow-up

Node 5b (Warm): [Parallel]
- Create Notion record
- Add to nurture sequence
- Send welcome email

Node 5c (Cold): [Single]
- Log and archive

Total nodes: 8
Execution time: ~3 seconds
Cost: ~$0.01 per lead
\end{lstlisting}
\end{codebox}

\section{Managing API Costs}

AI API costs can spiral quickly if you're not thoughtful. Here's how I keep them under control.

\subsection{Understanding Token Economics}

\begin{codebox}
\begin{lstlisting}[style=python]
CLAUDE API PRICING (as of 2026)

Model                    Input/1M    Output/1M
------------------------  --------   ---------
Claude 3.5 Sonnet        $3.00       $15.00
Claude 3.5 Haiku         $0.25       $1.25
Claude 3 Opus            $15.00      $75.00

TYPICAL COST PER TASK

Email triage (Haiku):
- Input: ~500 tokens = $0.000125
- Output: ~100 tokens = $0.000125
- Total: ~$0.00025 per email

Lead qualification (Sonnet):
- Input: ~1,000 tokens = $0.003
- Output: ~300 tokens = $0.0045
- Total: ~$0.008 per lead

Blog post (Sonnet):
- Input: ~2,000 tokens = $0.006
- Output: ~2,000 tokens = $0.03
- Total: ~$0.04 per post
\end{lstlisting}
\end{codebox}

\subsection{Cost Optimization Strategies}

\begin{codebox}
\begin{lstlisting}[style=python]
REDUCE COSTS BY 70%+

1. USE HAIKU FOR SIMPLE TASKS
   - Classification: Haiku
   - Entity extraction: Haiku
   - Simple Q&A: Haiku
   - Sonnet only for reasoning tasks
   - Savings: 90% vs using Sonnet for everything

2. CACHE REPEATED QUERIES
   - Same playbook = cached response
   - FAQ answers = cached
   - Simple Redis or file-based cache
   - Savings: 50%+ on repeated operations

3. BATCH OPERATIONS
   - Process emails in batches of 10
   - Daily summaries, not hourly
   - Fewer API calls = lower costs
   - Savings: 30% on overhead

4. TRUNCATE CONTEXT
   - Only send relevant history
   - Summarize long documents
   - Don't include entire customer file
   - Savings: 40% on token usage

5. PROMPT EFFICIENCY
   - Clear, concise instructions
   - Structured output (JSON)
   - Don't ask for explanations you won't use
   - Savings: 20% on verbosity
\end{lstlisting}
\end{codebox}

\subsection{Monthly Budget Examples}

\begin{codebox}
\begin{lstlisting}[style=python]
REALISTIC COST PROJECTIONS

Small Business (50 leads/mo, 100 emails/day):
- Email processing: 3,000/mo x $0.00025 = $0.75
- Lead qualification: 50/mo x $0.008 = $0.40
- Content (4 posts): 4 x $0.04 = $0.16
- Daily summaries: 30 x $0.03 = $0.90
- Support tickets: 50 x $0.01 = $0.50
- TOTAL: ~$3/month in API costs
- Add buffer: ~$10-15/month realistic

Medium Business (200 leads/mo, 300 emails/day):
- Email processing: 9,000/mo x $0.00025 = $2.25
- Lead qualification: 200/mo x $0.008 = $1.60
- Content (12 posts): 12 x $0.04 = $0.48
- Daily summaries: 30 x $0.03 = $0.90
- Support tickets: 200 x $0.01 = $2.00
- TOTAL: ~$7/month in API costs
- Add buffer: ~$25-35/month realistic

Growing Business (500 leads/mo, 500 emails/day):
- TOTAL: ~$15/month in API costs
- Add buffer: ~$50-75/month realistic

These costs are remarkably low.
The infrastructure overhead costs more than the AI.
\end{lstlisting}
\end{codebox}

\section{Security and Data Privacy}

This is where many founders get nervous---and rightfully so. You're sending customer data to AI providers. Here's how to do it safely.

\begin{codebox}
\begin{lstlisting}[style=python]
DATA SECURITY PRINCIPLES

1. MINIMIZE DATA SENT TO AI
   - Only what's needed for the task
   - Never passwords or API keys
   - Anonymize when possible
   - Reference IDs, not full records

2. USE ENTERPRISE TIERS
   - No training on your data (verify in ToS)
   - SOC 2 compliance
   - Data retention controls
   - Audit logs available

3. KEEP SENSITIVE DATA LOCAL
   - Full PII stays in your database
   - Financial details stay in your systems
   - Send only necessary context
   - "Customer #1234" not "John Smith SSN 123-45-6789"

4. AUDIT REGULARLY
   - Review what's being sent monthly
   - Check API logs for anomalies
   - Update playbooks as needed
   - Train any team members on protocols
\end{lstlisting}
\end{codebox}

\subsection{Compliance Checklist}

Before you deploy agents that handle customer data:

\begin{codebox}
\begin{lstlisting}[style=python]
PRIVACY COMPLIANCE

[ ] Data processing agreement with AI provider
[ ] Privacy policy updated for AI usage
[ ] Customer notification if required (GDPR)
[ ] Opt-out mechanism for AI processing
[ ] Data retention limits configured
[ ] Audit trail for AI-generated decisions
[ ] Human review process for critical decisions
[ ] Incident response plan for AI errors
\end{lstlisting}
\end{codebox}

\section{Your First Week}

Here's exactly how to get started, day by day.

\textbf{Days 1-2: Set Up Accounts}

\begin{codebox}
\begin{lstlisting}[style=python]
[ ] Claude Pro subscription ($20/mo)
[ ] n8n Cloud account ($20/mo) or self-host
[ ] Notion or Obsidian for knowledge base
[ ] Verify all accounts work
\end{lstlisting}
\end{codebox}

\textbf{Days 3-4: Your First Workflow}

\begin{codebox}
\begin{lstlisting}[style=python]
[ ] Connect email (Gmail API)
[ ] Create simple: Email -> Claude -> Draft Response
[ ] Test with your own emails
[ ] Refine the prompt based on results
\end{lstlisting}
\end{codebox}

\textbf{Days 5-7: Production Ready}

\begin{codebox}
\begin{lstlisting}[style=python]
[ ] Add error handling
[ ] Test edge cases
[ ] Document the workflow
[ ] Set up basic monitoring
[ ] Go live with supervision
\end{lstlisting}
\end{codebox}

\textbf{Week 2: First Real Agent}

Choose based on your biggest pain point:
\begin{itemize}
\item Drowning in email? Build Emma
\item Losing leads? Build Sam
\item Content inconsistent? Build Maya
\item Support taking forever? Build Casey
\end{itemize}

\begin{keyinsight}[The Stack Formula]
Your agent infrastructure should be boring. Reliable. Simple.

\textbf{Start Simple:} Claude Pro + n8n + Notion = \$50-100/month with eighty percent capability.

\textbf{Scale When Needed:} Add complexity only when simplicity fails.

\textbf{Core Patterns:} Every workflow is Event $\rightarrow$ AI $\rightarrow$ Action, just with different triggers and outputs.

The goal isn't the most sophisticated stack. The goal is the simplest stack that solves your problems. Remember my 2 AM infrastructure meltdown? That happened because I optimized for capability instead of reliability.

Build boring. Sleep well.
\end{keyinsight}

\textbf{Next Chapter:} Writing the prompts, playbooks, and protocols that make your agents consistent and reliable.
